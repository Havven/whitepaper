\section{Road Map}

The Havven system will be released in phases, described in the following
sections. This phased rollout accelerates market penetration while
allowing features to be introduced and monitored incrementally. The intended approach  
is to decentralise Havven as it is built, which will be accompanied by a gradual
increase in the issuance ratio while the system develops.
This provides a balance of safety and functionality until Havven is fully
mature.

\paragraph{Release Schedule}
\renewcommand{\arraystretch}{1.5}
\setlength{\tabcolsep}{13pt}
\begin{centering}
    \begin{tabular}{|c|l|c|}
        \hline
        \textbf{Type} & \textbf{Iteration} & \textbf{Release (2018)} \\
        \hline
        \hline
        eUSD & Ether-Backed Nomins & Q1 \\
        \hline
        \hline
        nUSD & System A: Static Foundation Issuance & Q2 \\ \cline{2-3}
        & System B: Universal Market Issuance & Q3 \\ \cline{2-3}
        & System C: Dynamic Fee Incentives & Q3 \\ \cline{2-3}
        & System D: Multi-Currency Nomins& Q4 \\
        \hline
    \end{tabular} \\
\end{centering}
\vspace{0.5cm}

\subsection{Ether-Backed Nomins (\texttt{eUSD})}

\noindent The ether-backed nomin system implements \texttt{eUSD}, an interim stablecoin
which is to operate while the havven-backed system is being developed.
Price stability is maintained by a pool of ether which backs the circulating stablecoin
supply. In this system, \texttt{eUSD} can be purchased from and sold into a pool
for \$1 worth of ether. The Havven foundation provides at least \$2 of ether collateral
for each issued \texttt{eUSD}, which ensures that the entire supply is redeemable for its
face value even in the face of the ether price falling by up to two thirds. Fees are
collected on transactions, and these fees are collected by havven owners. \\

\noindent The \texttt{eUSD} system, being based on ether collateral, has a constrained
maximum supply. Therefore, in order to scale, the system must move towards its ultimate
issuance mechanism, where nomins are backed with havvens.

\pagebreak
\subsection{Havven-Backed Nomins (\texttt{nUSD})}

\paragraph{System A} Static Foundation Issuance \\

\noindent The System A version of \texttt{nUSD} allows issuance up to a static
collateralisation ratio against havvens, which is set by the foundation.
\texttt{nUSD} will be issued directly into the issuer's wallet, and fees will be paid
proportionally with the number of issued nomins per user. Given that it's necessary to
encourage liquidity, but not all the mechanisms outlined in this paper will be operating
yet, issuance will be by the foundation itself, and potentially other white-listed
addresses it trusts. In this way, the stability of the token is maintained by direct
market intervention by the Havven foundation. \\

\noindent As System A fundamentally changes the issuance mechanics, \texttt{nUSD} is a distinct
token from \texttt{eUSD}, even though both target a price of \$1. At the time of the \texttt{nUSD}
launch, the \texttt{eUSD} contract will be destroyed after a liquidation period, when \texttt{eUSD}
can be sold back to the contract. Future updates to the Havven system will not entail the
destruction of any nomin tokens; thus both \texttt{havvens} and \texttt{nUSD} will persist
without further interruption. \\

\noindent System A limits issuance and fee returns to the Havven foundation itself. Therefore
only those havvens the foundation controls can be used to issue nomins against, which means
the full value of the havven network cannot be deployed for issuance. The fact that nomins
are created directly in the issuer's wallet may limit liquidity if issuers choose not to sell
those nomins. Systems B aims to combat such limitations. \\


\paragraph{System B} Universal Market Issuance \\

\noindent System B will enforce that issuance occurs through an issuance controller or
decentralised exchange to ensure that new liquidity is injected directly
into the market at \$1 per nomin. The Havven system will manage open
issue/burn orders while also continually updating the price at which they are
offered. \\

\noindent This phase will open issuance to the market at large, and fees will be rewarded
in proportion with the quantity of locked havvens. This opens up the opportunity
for anyone to engage in issuance. However, as not all of the intended incentive
mechanisms will be operational, the market will be monitored closely by the Havven
foundation, and the the issuance ratio maintained at a low level. These incentives will
be activated in System C. \\


\paragraph{System C} Dynamic Fee Incentives \\

\noindent The full featured incentive mechanisms will be activated so that users earn fees
in accordance with how effectively they stabilise the nomin price. This is the
system which is described in the main body of this paper. Once this version of the system
is operational, further extensions to its capabilities are anticipated for future work. \\


\paragraph{System D} Multi-Currency Nomins \\

\noindent In principle, the Havven mechanism can target any price. 
With this in mind, the foundation intends to allow issuance
of different flavours of nomins. For example, in addition to 
nUSD, the system could also allow nKRW to be issued, tracking
the price of the Korean won. In such a system, issuers would be
rewarded with fees only for currencies they have issued. This extension
would replicate the Havven mechanism for each currency. 
If such issuance occurs against the same collateral pool, it is
straightforward to see that in the absence of other incentives,
the issued supply of each nomin flavour should be proportional with the
each flavour's share of total transaction volume. \\

\noindent A multicurrency nomin system clearly has the advantage of providing the
benefits of stablecoins to many markets, and thus of deepening
the fee-generating volume for havvens. But that these nomin flavours
are interrelated by the common pool of capital provides extra utility
in the form of relatively cheap foreign exchange in nomins, using 
havven issuance to settle between them.
The introduction of new nomin flavours will become straightforward,
requiring only the creation of a new oracle for the intended asset,
which does not have to be a currency. \\


\subsection{Continuing Development}

\noindent The Havven protocol is not tied to any particular distributed ledger,
only requiring general smart contract capabilities and a modicum of speed.
It is, however, tied to the solution of a number of fundamental problems
in the blockchain space. To this end, research and development efforts
will continue in a number of relevant areas, which may include:

\begin{itemize}
    \item Improved fee structures, including dynamic fees and hedging charges.
    \item Fast decentralised oracles.
    \item Alternate blockchains and cross-chain functionality.
    \item Refined stabilisation mechanisms.
    \item System parameter optimisation.
    \item Stablecoin econometrics and modelling.
\end{itemize}

\noindent In addition to these fundamental questions, the Havven foundation will continue
to pursue integrations with projects which would benefit from a stable unit of
account. This encompasses any platform needing a settlement token which
integrates with decentralised systems. All such efforts aim to increase nomin
transaction volume, increasing the resources available for havven holders to
stabilise the token with.
