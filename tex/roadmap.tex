\section{Road Map}

The final Havven system will be released in phases. This appendix describes
each of these phases and their expected completion dates.

\paragraph{Ether-Backed Nomins} (\texttt{eUSD})

Duration: March until TBC1

Description:
The ether-backed nomin system implements \texttt{eUSD}, an interim stablecoin.
\texttt{eUSD} operates while the system is being developed.
The \texttt{eUSD} price's stability is maintained by a pool of ether which
backs the circulating stablecoin supply. In this system, \texttt{eUSD} can be
purchased from, and sold into, the pool for \(US\$1\) worth of ether.
The Havven foundation provides at least \(US\$2\) of ether collateral for each
issued \texttt{eUSD}, which ensures that the entire supply is redeemable for its
face value even in the face of the ether price falling by up to two thirds.
Fees are collected on \texttt{eUSD} transactions, and these fees are collected
by havven owners.

Features:
\begin{itemize}
    \item{\texttt{eUSD}}
    \item{havvens}
    \item{An ether price oracle}
\end{itemize}


\paragraph{System A} \texttt{nUSD}: Static C, proportional fees, and white-listed issuers

Duration: TBC1 until TBC2

Description:
The A version of the \texttt{nUSD} system will allow issuance up to a static
collateralisation ratio against havvens, which is set by the foundation.
\texttt{nUSD} will be issued directly into the issuer's wallet, and fees will be paid
proportionally with the number of issued nomins per user. Given that it's necessary to
encourage liquidity, but not all the mechanisms outlined in this paper will be operating
yet, issuance will be by the foundation itself, and potentially other white-listed
addresses it trusts.

The \texttt{eUSD} system, being based on ether collateral, has a constrained maximum supply.
Therefore, in order to scale, the system must move towards an issuance mechanism based on havven-backed
nomins. System A is the first step towards the final such system. As System A fundamentally
changes the issuance mechanics, \texttt{nUSD} will be a distinct token from \texttt{eUSD},
with \texttt{eUSD} exchangeable one-for-one with \texttt{nUSD} through the Havven
foundation at the time of the \texttt{nUSD} launch, or exchangeable for \(US\$1\) worth of ether,
as usual. After a liquidation period, the \texttt{eUSD} contract will be destroyed.
Future updates to the Havven system will not entail the destruction of any nomin tokens; thus
both \texttt{havvens} and \texttt{nUSD} will persist without further interruption.

Features:
\begin{itemize}
    \item{\texttt{nUSD}}
    \item{Havven-backed issuance mechanics}
    \item{A havven price oracle}
\end{itemize}


\paragraph{System B} \texttt{nUSD}: Universal non-discretionary issuance

Duration: TBC2 until TBC3

Description:

System B will enforce that issuance occurs through a DEX so as to ensure that
new liquidity is actually injected into the market. The Havven system
will have to manage the volume of issue/burn orders on either side of the book
while also continually updating the price at which they are offered.


\paragraph{System C} \texttt{nUSD}: Dynamic fee computations

Duration: TBC3 until TBC4

Description:

Features:


\paragraph{System D} Multi-currency nomins

Duration: TBC4 until TBC5

Description:

Features:
