\section{Alternative approaches}

\todo[inline]{Establish a summary of arguments against each competitor}
\todo[inline]{Makerdao critique.}
\subsection{Basecoin}

\paragraph{Description of system}

\noindent Basecoin is described as operating similarly to Havven in that there is separation between a backing token and a transactional token, however Basecoin also separates out a specific ``bond'' token.
The peg to an arbitrary external asset is maintained by using an oracle service to discover the price on an external market, before regulating the supply of ``basecoins'' through actively increasing (issuing new basecoin),
and decreasing (auctioning of bonds) the supply, effectively acting as an autonomous central bank.

\subsubsection{Key issues}

\noindent Basecoin is intended to operate ``as a decentralized, protocol-enforced algorithm, without the need for direct human judgment (sic). For this reason,
Basecoin can be understood as implementing an algorithmic central bank.'' Whilst not without merit, this approach was discarded by Havven due to the high degree
of design complexity required to be anticipated in order to ensure the stabilisation mechanism is effective. The paper states that Monte Carlo simulations have
been run which indicate stability under a range of scenarios, however details are yet to be released by the team. \\

\noindent Another element not explored in the Basecoin whitepaper is the incentives for participants to engage with the cryptoeconomic system itself.
While there is no argument against the utility of stablecoins, there must be incentives inherent in all such systems to ensure the appropriate
participation of all actors. In this case, there are consumers of the stablecoin and active participants in the monetary policy. It is critical
to be able to demonstrate that the incentives within the system will ensure profitable participation strategies for actors. Without this being clarified
it is unclear as to whether there will be uptake by enough users to generate sufficient currency in circulation to support the demand for a stablecoin.
Critically, the removal of Basecoin from the system to ensure the stable peg is predicated on the significant assumption that participants will take positions
in the ongoing bond auctions. This assumption remains untested. \\

\noindent A final point needs to be made with respect to the overarching monetary approach
espoused in the whitepaper. In the section ``Averting Macroeconomic Depressions'' the authors appear to support
money printing and inflationary policies and the subsequent devaluation of currency. Even were it possible to
demonstrate that inflation of the money supply via such a system would be effective in combating a deflationary spiral,
a far better argument could be made that simply by implementing a stable store of value and unit of account that such a
system would not be required. Generally, the apparent assumption that such a system would be achievable and still able to
handle monetary crises in a far future time without centralised intervention stretches credulity. It's not entirely clear
why Basecoin has intended to merely replicate the function of a central bank, rather than aim for pure stability or a
relative-stable approach such as Havven. We are skeptical of any group that would advocate for monetary approaches that are
diametrically opposed to cryptoeconomic efforts to democratise money, and we feel that the proposal to intentionally
create a systematically inflationary monetary system is not the answer. Instead, we should at this point in time be
aiming to construct a system that provides a stable store of value relative to an arbitrary fiat currency. The macroeconomic
benefits of such a system are clear, and for as long as we live in a fiat-dominated world this will continue to be the case.

\subsection{Tether}

\paragraph{Description of system}

Tethers accepts fiat deposits into the Hong Kong-based Tether Limited bank account and issues ``USDT'' (USD Tether) over Bitcoin via the Omni Layer protocol. Tethers are an asset-backed digital token, representing a claim on the cash held in reserve. \\

\noindent The stability of the USDT `coin' effectively relies on the force of external market arbitrage to ensure the peg holds over time.

\paragraph{Key issues}

Despite the whitepaper claiming that the ``goal of any successful cryptocurrency is to completely eliminate the requirement for trust,'' and that each Tether is ``fully redeemable/exchangeable any time for the underlying fiat currency,'' the company's terms of service quite clearly state that ``there is no contractual right or other right or legal claim against us to redeem or exchange your Tethers for money.'' \\

\noindent Tether clearly relies on a manual, centralised proof of existence for the backing asset, and so suffers from the very issue that the Tether whitepaper decries. Indeed the same issue is encountered with tokenised gold, or similarly any other real-world asset where some Oracle bridge is required to interface into a distributed ledger.

\paragraph{Current state}

Recently, Tether announced support for issuing ERC-20 compatible tokens on Ethereum as opposed to releasing ``tethers'' on the Bitcoin blockchain using the Omni Layer protocol. \\

\noindent At the time of writing, the market capitalisation for USDT was approximately \$440m, and the discrepancy regarding their terms of service remains unresolved. \\


% \subsection{MakerDAO}

% \paragraph{Description of system}

% \paragraph{Key issues}

% \paragraph{Current state}

% \subsection{Nubits}

% \paragraph{Description of system}

% \paragraph{Key issues}

% \paragraph{Current state}

\pagebreak