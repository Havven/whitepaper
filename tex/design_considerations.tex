\section{Design Considerations}

Havven works by providing a set of market incentives that support the stability of nomin value with respect to an external asset.

% \todo[inline]{Address how to maintain baseline demand.}
% \todo[inline]{Address the role of the Havven foundation.}
% \todo[inline]{Outline basic verbs for market players.}

% \todo[inline, caption={Functional description refactor.}]{Move analytical subsections from functional description section to the qualitative analysis and/or rationale sections.}
\subsection{Investment incentives}

We consider the reasons why any rational actor would buy havvens. A potential buyer has at least three avenues for making money in Havven:

\paragraph{Capital gains due to the appreciation of havvens:}
Due to its constrained supply, and the intrinsic utility of the stablecoin that it backs, it's reasonable to assume that
havvens will appreciate in price.

\paragraph{Interest accrued from fees:}
If the price of havvens stabilises for long periods of time, fees may be the only source of revenue. Ideally fees are set at a level where they are both high enough to be an incentive for rent-seekers to hold havvens in the long term (thus assuming the risk of providing collateral for the system) and low enough not to be a disincentive for ordinary users to transact in nomins.
It is desirable, perhaps in a future world dominated by micropayments, for these fees to be negligible for end users, while still being macroeconomically important for the system, and for those who capitalise it.

\paragraph{Arbitrage profit:}
It is the arbitrageurs who will ultimately bring the price of nomins back into balance by a triangular circuit through nomins, havvens, and the external (crypto or fiat) markets. Arbitrageurs might hold havvens for a short time in order to pursue this strategy.


\pagebreak
\subsection{Fees}

There are several key considerations with respect to fee design:

\subsubsection{Fee design considerations}

\paragraph{The purpose of fees}

Fees are intended to be redistributed to actors who support the stability of the system. A fee pool will be distributed periodically for this purpose.
If the system determines that the nomin price is too low, then fees could be burned. If the price is too high then the system could sell these back
into the system at a discounted rate. The fee collection rate will also be a direct measure of the velocity of money in Havven. It's in the interest
of havven holders to maximise liquidity in order to maximise their return.

\paragraph{Fee beneficiaries}

One possibility would be simply to award fees to any holder of havvens,
but in this situation holders can get all the benefit without taking any risk.
Although in the aggregate, it would be better for havven-holders if everyone issued nomins,
the marginal return for any single player (who cannot issue a large fraction of all circulating nomins)
of actually issuing them would not outweigh the risk they take on in doing so. If a user can issue 1\% of circulating
nomins, then doing so will only increase their fee takings by 1\%. Hence rational actors would not be incentivised to issue nomins at all.
This is a classic tragedy of the commons.

\noindent In order to avoid this situation, we must improve the marginal benefit of issuing nomins into circulation.
Hence, fees must be paid to those who \textit{issue} nomins, not just those who hold havvens.

\paragraph{Fee collection}

The system can charge fees whenever any value is transferred, or any state is updated.
\noindent Different fee rates have different macroeconomic effects. We might in general like to set higher havven than nomin transfer fees, making the stablecoin itself a lower friction market in order to incentivise its use for exchange. Meanwhile, issuance and redemption fees will change the difficulty of entering and exiting the issuance game. \\

\noindent It is also possible for fees to float. The fee schedule could be altered dynamically in order to stabilise the system. It is even conceivable that the system could set negative fee rates if it needed to and charge punitive fees if a user is above the targeted utilisation ratio. For example, if nomin liquidity is low, meaning the system wants to incentivise issuance, then nomin transfer fees could increase, thus having the combined effect of increasing the interest accrued by issuers (thus incentivising issuance) and at the same time making it more expensive to transact in nomins. This would reduce demand and decrease the liquidity requirements. \\

\noindent Of note, fees are antithetical to arbitrage. The higher the fee, the higher the transaction friction, and the harder it is to make money by arbitrage. For example, if exchange fees amount to 1\% per trade, then a full arbitrage cycle between all three markets, (nomins, havvens, and fiat) will cost in excess of 3\%. So it would not make sense to undertake arbitrage until such a time as the quoted exchange rate is misvalued by more than 3\% relative to the cross exchange rate. Hence, fees compete with arbitrage to stabilise price. Lower fees allow tighter stabilisation, within a window exactly in proportion with the fee rates themselves.

\subsection{Encouraging liquidity}

\noindent It's desirable that when actors issue nomins they are actually injected into the liquidity pool for their intended use,
rather than being held by the same actor in order to benefit from both the receipt of fees while retaining the option of using those nomins to rapidly release their havvens.
In this manner they would accrue fees, but take on none of the risk of spending those nomins, for they always have an instantaneous option to liquidate their position and escape.
On the other hand, an actor who had done the economically-desirable thing and issued nomins to the market would be forced to buy them back before redeeming their escrowed havvens.

\subsubsection{Non-discretionary Issuance}

One possibility is to simply provide an issuer no control over the tokens they issue. That is, when a quantity of nomins is issued, they are generated by the system, which then places a sell order at the current going rate for that quantity on an exchange on the behalf of the issuer. When the order is filled, the proceeds in ether are remitted to the issuer. \\

\noindent Conversely, when a quantity of nomins is burned, they must first be obtained from the open market. In this way, a user would indicate an intention to burn, providing sufficient value to buy the proposed quantity of nomins, and the system would bid for that quantity on their behalf, thereby liquidating the user's havven position once the nomins have been obtained. \\

\noindent So one might consider there to be a formal distinction between wallets that issue tokens and those that do not. In this vein, one might envisage an extra fee to be charged to directly transfer nomins (rather than buying from the market) into a wallet that has an outstanding quantity of nomins it has previously issued, but not burnt. The result of this is that it would be less reasonable for an agent to sit on nomins in order to burn them in future as it is more advantageous in times of relative stability to simply buy them from the market. \\

\pagebreak 

\subsection{Price discovery}

One of the key challenges with denominating a cryptocurrency in a fiat currency is the
fundamental link this creates to the centralised world. When the denominating currency exists
external to the blockchain ecosystem, some bridge must be built so that the system can act with
knowledge of external information.
We recognise that a decentralised price-finding mechanism would be our preferred approach.
Research into this mechanism is on our horizon, and future versions of this white paper will contain
our results.
However, in order to reclaim system performance, we can trade some of the trustlessness of the design, for example by
a trusted ``Oracle'' service, which transmits knowledge of the external world into the system, building a causal link.


\subsection{Utilisation Ratio} Even though rational actor modelling suggests that the price of nomins and havvens will equilibrilate
given that an agent may pay up to some multiple of the market value of a nomin in order to release escrowed havvens, we are aware that there may be some
prevailing macroeconomic or psychological influences relating to an undercollateralised position
(i.e. if the value of the collateral pool is less than the issued stablecoin). As such, our modelling incorporates the notion of a ``utilisation ratio''
\(0 < U < 1\), such that the system is over-collateralised in an attempt to counteract these potential issues. It may be that resolving an
optimised utilisation ratio is beyond the ability of our agent-based modelling to determine, and as such, selecting this may need to be
informed by the activity of a live system. Thus it is currently intended for Havven to initially include in its governance model the power
to correct the utilisation ratio. This power can be removed over time as the system is proven, perhaps directly linked to some parametric milestones such as nomin velocity and stability.

%\paragraph{Direct Redemption} Direct redemption is a system design option to allow a holder of nomins to redeem any escrowed havvens. This option allows more efficient redemption of nomins for the backing value, however introduces the difficulty of liquidating another actor's escrowed havvens and interrupting their collection of fees. This can be solved by adding a premium to the price of an escrowed havven (potentially user-defined) over the current value. This premium would need to be paid in ether as to preserve the symmetric issuance and destruction of nomins. \\

\pagebreak
