\section{System Description} Havven is a dual-token system that, combined
with a set of novel incentive mechanisms, stabilises the price of the nomin
with respect to an external asset. Users of the nomin token pay the owners of
the havven token for collateralising and stabilising the system. \\

\noindent The havven token incentivises those who hold it to fulfil two functions:

\begin{itemize}
\item{To provide the system with collateral.}
\item{To participate in the stabilisation of the nomin price.} \\
\end{itemize}

\noindent \textbf{Collateralisation} \\

\noindent Confidence in the stability of the nomin begins with
overcollateralisation, so that the value of escrowed havvens is greater than
the value of nomins in circulation. As long as the ratio of total nomin value
to total havven value remains favourable, there is sufficient backing in the
underlying collateral pool to ensure that nomins can be redeemed for their
face value. The redeemability of a nomin for the havvens against which it was
issued strongly supports a stable price. \\

\noindent \textbf{Stabilisation Incentives} \\

\noindent Havven rewards those that have issued nomins. These rewards are
derived from transaction fees and are distributed in proportion with how
effectively each issuer acts to maintain the correct nomin supply.
The system monitors the nomin price, and responds by adjusting its targeted
global supply, which individual issuers are incentivised to move towards. \\

\noindent Where volatility persists, stronger stabilisation mechanisms may be
applied, for example automated collateral recovery. Where a significant
portion of nomins are being used for hedging, (and hence not generating
transaction fees) a charge can be applied to ensure that the cost of utility
for hedging is not being solely borne by transactions.

\newpage

\subsection{Equilibrium Nomin Price}

\noindent We first introduce the core system variables:

\begin{align*}
H &\text{\ \ havven quantity} & N &\text{\ \ nomin quantity} \\
P_h &\text{\ \ havven price}  & P_n &\text{\ \ nomin price} \\
\end{align*}

% \todo[inline]{Consider renaming havven price symbol to reflect that it is computed from income}

\noindent All havven tokens are created at initialisation, so \(H\) is
constant. The quantity of nomins floats, responding to the issuance actions
of havven holders. The Havven system needs to incentivise issuers to maintain
\(N\) such that the nomin price \(P_n\), is stable at \$1. As we proceed, we may
subscript variables with \(t\) to indicate the value of that variable at a
given time. Any variable lacking such a subscript indicates the value of the
quantity it represents at the current time. \\

\noindent In Havven, the measure of the value of nomins against the value of
havvens is called the collateralisation ratio:

\begin{equation}
C \ = \ \frac{P_n \cdot N}{P_h \cdot H} \label{eq:collateralisation}
\end{equation}

\vspace{3 mm}

\noindent From the law of supply and demand, there exists some supply of
nomins \(N_{opt}\), where the related level of demand yields an equilibrium
price of \$1. This quantity is associated with an optimal collateralisation
ratio \(C_{opt}\). We visualise this equilibrium below with a hypothetical
demand and supply curve. \\

% \todo[inline]{curved lines}
% \todo[inline]{make diagrams pretty}
\begin{center}
\begin{tikzpicture}[scale=3]
    % draw axes
    \draw [<->, thick] (0,2) node (yaxis) [above] {\(P_n\)} |- (2.5,0) node (xaxis) [right] {\(N\)};
    % draw intersecting lines
    \draw (0.5, 0.5) coordinate (a_1) -- (2,1.8) coordinate (a_2) node[pos=0.0, left] {S};
    \draw (0.5, 1.8) coordinate (b_1) -- (2,0.5) coordinate (b_2) node[pos=1.0, right] {D};
    % calculate coordinate of intersection
    \coordinate (c) at (intersection of a_1--a_2 and b_1--b_2);
    \draw[dashed] (yaxis |- c) node [left] {\(\$1\)} -| (xaxis -| c) node[below] {\(N_{opt} = C_{opt} \cdot P_h \cdot H\)};
\end{tikzpicture}
\end{center}

\noindent The system is unable to influence the demand for nomins. We assume
that some level of demand exists given the utility of nomins as a stable
cryptocurrency. Although demand cannot be manipulated, the supply of nomins
is controlled by havven holders, whose issuance incentives are in turn
controlled by the system. It follows that as we require a fixed price \(P_n =
\$1 \) and are unable to control either \(P_h\) or \(H\), we must manipulate
\(C_{opt}\) such that \(N = N_{opt}\) in order to satisfy our requirement.

\subsection{Issuance and Collateralisation} 

\noindent Havven's goal is to stabilise the nomin price while remaining
overcollateralised. In order to do so, the system defines a collateralisation target:

\begin{equation}
0 < C_{opt} < 1  \label{eq:target}
\end{equation}

\vspace{2 mm}

\noindent \(C_{opt}\) is implicitly closer to \(C^*_{P_h}\) than \(C\) is; when \(C\)
approaches \(C_{opt}\), it also approaches \(C^*_{P_h}\). \\

\noindent It is necessary at this point to distinguish for an account \(i\)
the nomins it contains \(N_i\) (equity) from the nomins it has issued
\(\check{N_i}\) (debt). Note that globally, \(\sum_{i}N_i = \sum_{i}\check{N_i}\),
as all nomins were issued by some account. However, a given account may have a
balance different from its issuance debt. Hence we can define the collateralisation
ratio for an individual account \(i\) in terms of its issuance debt:

\begin{equation}
C_i \ = \ \frac{P_n \cdot \check{N_i}}{P_h \cdot H_i}  \label{eq:individualcollateralisation}
\end{equation}

\vspace{2 mm}

\noindent The system provides incentives for individual issuers to bring
their \(C_i\) closer to \(C_{opt}\) while maintaining \(C_{opt}\) itself at a level
that stabilises the price. As each \(C_i\) targets \(C_{opt}\), the overall
collateralisation ratio does also. \\

\noindent \textbf{Nomin Issuance}

\vspace{1mm}

\noindent The nomin issuance mechanism allows Havven to reach its
collateralisation target. Issuing nomins escrows some quantity of havvens,
which cannot be moved until they are unescrowed. The quantity of havvens
\(\check{H_i}\) locked in generating \(\check{N_i}\) nomins is:

\begin{equation}
\check{H_i} \ = \ \frac{P_n \cdot \check{N_i}}{P_h \cdot C_{max}}  \label{eq:escrowed}
\end{equation}

\vspace{2 mm}

\noindent Under equilibrium conditions, there is some \(\check{H_i} \leq H_i\)
when \(C_i\) coincides with \(C_{opt}\), which the issuer is incentivised to
target. These incentives are provided in the form of transaction fees,
discussed in section 2.4. It is important to note that the issuer may
voluntarily increase their \(C_i\) up to the limit of \(C_{max}\); for example if
they anticipate a positive movement in \(C_{opt}\). \\

\noindent On the other hand, no one may issue a quantity of nomins
that would lock more than \(H_i\) havvens. Consequently, \(C_i\) may never exceed
\(C_{max}\), except by price fluctuations, and in such circumstances, issuers
are rewarded for bringing \(C_i\) back under \(C_{max}\). \\

\noindent After generating the nomins, the system places a \textbf{limit
sell} order with a price of \$1 on a decentralised exchange. This means that
the nomins will be sold at the current market price, down to a minimum price
of \$1. If we assume implementation on Ethereum, then the nomins are sold for
an equivalent value in ether, with the proceeds of the sale remitted to the
issuer. \\

\noindent \textbf{Nomin Destruction}

\vspace{1mm}

\noindent In order to access the havvens that have been escrowed, the system
must destroy the same number of nomins that were originally issued. When the
issuer indicates the intention to retrieve their havvens, the system places a
\textbf{limit buy} order on a decentralised exchange, up to a maximum price
of \$1. The system places this order on behalf of the issuer and upon
completion, the nomins are immediately destroyed. \\

\newpage
\subsection{Transaction Fees}
Havven needs a direct incentive mechanism that can correct the global
collateralisation ratio, \(C\), in response to changes in the price of havvens
or nomins. In order to target the correct price, the system adjusts the fees
it pays to havven holders as according to their effectiveness in stabilising
the price.

\subsubsection{Fees Received by Havven Holders}

\noindent The fee rate received by a havven holder that has escrowed havvens is
\(\phi_{r,i}\). The actual fee they receive is \(H_i \cdot \phi_{r,i}\), being
proportional with their havven balance. This rate changes with respect to
their unique collateralisation ratio, \(C_i\). It increases linearly to a
maximum \(\phi_{base}\) at the optimal collateralisation ratio \(C_{opt}\),
before quickly diminishing as \(C_i\) approaches the maximum collateralisation
ratio \(C_{max}\). \\

\noindent This function is designed to encourage havven holders to constantly
target the optimal collateralisation ratio, by rewarding them with greater
fees if they bring their \(C_i\) into alignment with \(C_{opt}\).\\

\begin{equation}
\phi_{r,i} \ = \ \phi_{base} \cdot \mathit{\Gamma}_{i}  \label{eq:feesreceived}
\end{equation}

\begin{equation}
\mathit{\Gamma}_{i} \ = \
\begin{cases}
 \frac{C_{i}}{C_{opt}} &\mbox{when } C_{i} \leq C_{opt} \\[1em]
 \frac{C_{max} - C_{i}}{C_{max} - C_{opt}} &\mbox{when } C_{opt} < C_{i} \leq C_{max} \\[1em]
 0 &\mbox{otherwise}
 \end{cases}
 \label{eq:7}
\end{equation}

\begin{center}
\begin{tikzpicture}[scale=3]
    % draw axes
    \draw [<->, thick] (0,2) node (yaxis) [above] {\(\phi_{r,i}\)} |- (2.5,0) node (xaxis) [right] {\(C_i\)};
    % draw two lines
    \draw (0.0, 0.0) coordinate (a_1) -- (1.75,0.75) coordinate (a_2);
    \draw (1.75, 0.75) coordinate (a_1) -- (2.25,0.0) coordinate (a_2);
    \coordinate (c) at (1.75, 0.75);
    \coordinate (d) at (2.25, 0.0);
    \coordinate (e) at (0.00, 0.00);
    \draw[dashed] (yaxis |- c) node [left] {\(\phi_{base}\)} -| (xaxis -| c) node [below] {\(C_{opt}\)};
    \draw[dashed] (xaxis -| d) node [below] {\(C_{max}\)};
\end{tikzpicture}
\end{center}

\newpage

\subsubsection{Nomin Transaction Fees}
Every time a nomin transaction occurs, the Havven system charges a small
transaction fee. Transaction fees allow the system to generate revenue, which
it can distribute to havven holders as an incentive to maintain nomin supply
at \(N_{opt}\). \\

\noindent The fee rate charged on nomin transactions is \(\phi_c\). It is
constant and will be sufficiently small that it provides little to no
friction for the user. We may then express the total fees collected in the
last period, \(F\), as a function of the velocity of nomins \(v\) and the total
nomin supply \(N\):

\begin{equation}
    F \ = \ v \cdot \phi_c \cdot N
\end{equation}

\subsubsection{Base Fee Rate}

Let us define the total fees received by havven holders \(F_{r}\): \\

\begin{equation}
F_{r} \ = \ \sum_{i} H_{i} \cdot \phi_{r,i} \label{eq:totalfeesreceived}
\end{equation} \\

\noindent Havven requires that the total fees collected from users has to be
equal to the total amount of fees paid to the havven holders, so that \(F_{r}
= F\). Substituting our earlier definition~\eqref{eq:feesreceived} for
\(\phi_{r,i}\) and solving for \(\phi_{base} \): \\

\begin{equation}
\phi_{base} \ = \ \frac{F}{\sum_{i} H_{i} \cdot \mathit{\Gamma}_{i}} \label{eq:10}
\end{equation} \\

\noindent We have now defined the maximum fee rate, \(\phi_{base}\), in terms
of the fees collected, \(F\). This rate should be achieved when an individual's
\(C_i\) is at \(C_{opt}\). \\

\noindent The definition of \(C_{opt}\) must therefore provide the following
incentive. If \(P_n > \$1\) then the system must encourage more nomins to be
issued. However, if \(P_n < \$1\), the system must encourage nomins to be
burned.

\newpage
\subsection{Collateralisation Ratio}
\subsubsection{Optimal Collateralisation Ratio}

\noindent The optimal collateralisation ratio \(C_{opt}\) is a target for
havven holders to reach in order to maximise the amount of fees they receive.
\(C_{opt}\) is defined in terms of the nomin price \(P_n\), such that its value
directly tracks changes in the nomin price; a havven holder wishing to
maximise their fees will target \(C_{opt}\) by issuing or destroying nomins. \\

\noindent The function for \(C_{opt}\) given below provides our dynamic target
for havven holders based on the price of nomins:

\begin{gather} \label{eq:optcollateralisation}
\begin{align}
\begin{split}
C_{opt} \ &= \ f(P_{n}) \cdot C  \\ 
f(P_n) \ &= \ \max(\sigma \cdot {(P_n - 1)}^{\psi} + 1, 0) \\
\sigma \ & \text{ \ price sensitivity parameter } (\sigma > 0)\\
\psi   \ & \text{ \ flattening parameter } (\psi \in \mathbb{N} \text{, } \psi \nmid 2) \\
\end{split}
\end{align}
\end{gather}


\begin{center}
\begin{tikzpicture}[scale=1.15]
\begin{axis}[
    axis lines = left,
    xlabel = \(P_n\),
    ylabel = {\(f(P_n)\)},
    xtick = {1},
    ytick = {1},
    y label style={at={(axis description cs:.15,1.00)}, rotate=270,anchor=south},
    x label style={at={(axis description cs:1.05,.13)},anchor=north},
]
\addplot[domain=0:2, range=0:2]{(x-1)^3 + 1};
\addplot[domain=0:2, range=0:2, dashed]{x};
\addplot[dashed, samples=50, smooth,domain=0:6] coordinates {(1,1)(1,0)};
\end{axis}
\end{tikzpicture}
\end{center}

\noindent When \(P_n\) is at \$1, \(C_{opt} = C\) and there is no incentive given
to move away from the current global collateralisation level. However, if
\(P_n < \$1\), then \(C_{opt} < C\), incentivising issuers to burn nomins,
thereby raising the price. The \(P_n > \$1\) case is symmetric. Notice that
when the \(P_n\) is close to \$1, \( f'(P_n) \) is small. However, the further it
diverges from \$1, the larger the slope becomes, providing a stronger
incentive, in the form of potential fees, for a havven holder to move toward
\(C_{opt}\). \\

\newpage

\subsubsection{Maximum Collateralisation Ratio}

\noindent Havven seeks to maintain \(C \leq C_{opt} < C_{max} < 1\), in order
to retain sufficient overcollateralisation. It might seem intuitive that
\(C_{max}\) should be a static value. However, since \(C_{opt}\) varies linearly
with \(P_n\) and inversely with \(P_h\), there are situations where \(C_{max}\) may
need to change. Below we define \(C_{max}\) in terms of \(C_{opt}\). \\

\begin{gather} \label{eq:maxcollateralisation}
\begin{align}
\begin{split}
C_{max} \ &= \ a \cdot C_{opt} \\ 
a & \geq 1 \\
\end{split}
\end{align}
\end{gather}

\begin{center}
\begin{tikzpicture}[scale=1.15]
\begin{axis}[
    axis lines = left,
    xlabel = \(C_{opt}\),
    ylabel = \(C_{max}\),
    xtick = {1},
    ytick = {1},
    xticklabels = {,,},
    yticklabels = {,,},
    xtick style = {draw=none},
    ytick style = {draw=none},
    y label style={at={(axis description cs:0.18,1.00)}, rotate=270, anchor=south},
    x label style={at={(axis description cs:1.08,0.14)}, anchor=north},
]
\addplot[domain=0:0.8]{1.25 * x};
\addplot[domain=0:1, dashed, smooth]{x};
\end{axis}
\end{tikzpicture}
\end{center}

\noindent The value of \(C_{max}\) determines how overcollateralised each nomin
is at issuance. The higher its value, the more nomins can be generated for
the same quantity of havvens. In contrast, if \(C_{max}\) is low, the system
has a greater capacity to absorb negative shocks in the havven price before
it becomes undercollateralised. The value of \(C_{max}\) therefore represents a
tradeoff between \textit{efficiency} and \textit{resilience}. By defining
\(C_{max}\) as a function of \(C_{opt}\), Havven finds the optimal balance in
this dilemma. This ensures that like \(C_{opt}\), \(C_{max}\) only changes as a
consequence of instability in the nomin price. \\

\noindent It should be noted that \(C_{max} > 1\) corresponds to a fractional
reserve monetary system, where a greater value of nomins can be issued
against each havven. In Havven, this situation is unsustainable because it
would cause simultaneous appreciation of havvens (up to at least the value of
nomins issuable against a havven) and depreciation of nomins, immediately
diminishing \(C\), \(C_{opt}\) and \(C_{max}\), bringing them back under \(1\).

