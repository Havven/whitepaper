\newpage

\subsection{Fee Evasion}

\noindent Being implemented with smart contracts, Havven is potentially
vulnerable to its tokens being wrapped by another smart contract which takes
deposits, and replicates all exchange functionality on redeemable IOU tokens
it issues. These wrapped tokens could then be exchanged without incurring fees.
We consider this situation unlikely for a number of reasons. \\

\noindent First, the fees are designed to be low enough that most users
shouldn't notice them, so they will not in general be strongly motivated to
switch to a marginal and less trustworthy alternative. Additionally, transfer
fees will still be charged upon deposit into and withdrawal from
token-wrapping contracts, which partly constrains the utility of the wrapper.

\noindent Second, network effects are tremendously important for currencies;
in order to have utility a token must be accepted for exchange in the
marketplace. This is challenging enough in itself, but a wrapped token must
do this to the extent that it displaces its own perfect substitute: the token
it wraps.
In Havven's case, this would undermine its built-in
stabilisation mechanisms, which become more powerful with increased
transaction volume. Consequently, as a wrapped nomin parasitises more of the
nomin market, it destroys the basis of its own utility, which is that nomins
themselves are stable.

\noindent Finally, it is unlikely that a token wrapper will be credible, not
having been publicly and expensively audited, while its primary function
undermines the trustworthiness of its authors. \\

\noindent Even as token wrapping may appear unlikely to the authors, there
are at least two remedies which can be instituted to resolve this. \\

\noindent It would be a simple matter to implement a democratic remedy,
weighted by havven balance, by which havven holders can freeze or confiscate the
balance of any contract that wraps assets. Those havven holders are
incentivised not to abuse this system for the same reason that bitcoin
mining pools do not form cartels to double-spend: because abuse of this
power would undermine the value of the system, and thus devalue their
own holdings.

\noindent The credible threat of such a system existing is enough to
discourage token wrappers from being used, even if they are written, since
any user who does so risks losing their entire wrapped balance.

\noindent An alternative solution is to institute a hedging fee, charged on static nomin
balances. Such a fee could be discounted against transfer fees so that it
encourages general token velocity, and imposes a cost on those who buy nomins
simply in order to hedge, and are thus a risk for the network. Under this
model, a token wrapper would not dodge this fee: the full balance that was
wrapped will not be available at the time of unwrapping.
