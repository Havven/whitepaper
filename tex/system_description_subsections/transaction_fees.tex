\subsection{Transaction Fees}
Havven needs a direct incentive mechanism that can correct the global
collateralisation ratio, \(C\), in response to changes in the price of havvens
or nomins. In order to target the correct price, the system adjusts the fees
it pays to havven holders according to their effectiveness in stabilising
the nomin price.

\subsubsection{Fees Received by Havven Holders}

\noindent The fee rate received by a havven holder that has escrowed havvens is
\(\phi_{r,i}\). The actual fee they receive is \(H_i \cdot \phi_{r,i}\), being
proportional with their havven balance. This rate changes with respect to
their unique collateralisation ratio, \(C_i\). It increases linearly to a
maximum \(\phi_{base}\) at the optimal collateralisation ratio \(C_{opt}\),
before quickly diminishing as \(C_i\) approaches the maximum collateralisation
ratio \(C_{max}\). \\

\noindent This function is designed to encourage havven holders to constantly
target the optimal collateralisation ratio, by rewarding them with greater
fees if they bring their \(C_i\) into alignment with \(C_{opt}\).\\

\begin{equation}
\phi_{r,i} \ = \ \phi_{base} \cdot \mathit{\Gamma}_{i}  \label{eq:feesreceived}
\end{equation}

\begin{equation}
\mathit{\Gamma}_{i} \ = \
\begin{cases}
 \frac{C_{i}}{C_{opt}} &\mbox{when } C_{i} \leq C_{opt} \\[1em]
 \frac{C_{max} - C_{i}}{C_{max} - C_{opt}} &\mbox{when } C_{opt} < C_{i} \leq C_{max} \\[1em]
 0 &\mbox{otherwise}
 \end{cases}
 \label{eq:7}
\end{equation}

\begin{center}
\begin{tikzpicture}[scale=3]
    % draw axes
    \draw [<->, thick] (0,2) node (yaxis) [above] {\(\phi_{r,i}\)} |- (2.5,0) node (xaxis) [right] {\(C_i\)};
    % draw two lines
    \draw (0.0, 0.0) coordinate (a_1) -- (1.75,0.75) coordinate (a_2);
    \draw (1.75, 0.75) coordinate (a_1) -- (2.25,0.0) coordinate (a_2);
    \coordinate (c) at (1.75, 0.75);
    \coordinate (d) at (2.25, 0.0);
    \coordinate (e) at (0.00, 0.00);
    \draw[dashed] (yaxis |- c) node [left] {\(\phi_{base}\)} -| (xaxis -| c) node [below] {\(C_{opt}\)};
    \draw[dashed] (xaxis -| d) node [below] {\(C_{max}\)};
\end{tikzpicture}
\end{center}

\newpage

\subsubsection{Nomin Transaction Fees}
Every time a nomin transaction occurs, the Havven system charges a small
transaction fee. Transaction fees allow the system to generate revenue, which
it can distribute to havven holders as an incentive to move the circulating nomin supply
towards \(N^*\). \\

\noindent The fee rate charged on nomin transactions is \(\phi_c\). It is
constant and will be sufficiently small that it provides little to no
friction for the user. We may then express the total fees collected in a period,
\(F\), as a function of the velocity of nomins \(v\) and the total
nomin supply \(N\):

\begin{equation}
    F \ = \ v \cdot \phi_c \cdot N
\end{equation}

\subsubsection{Base Fee Rate}

Let us define the total fees received by havven holders \(F_{r}\): \\

\begin{equation}
F_{r} \ = \ \sum_{i} H_{i} \cdot \phi_{r,i} \label{eq:totalfeesreceived}
\end{equation} \\

\noindent Havven requires that the total fees collected from users has to be
equal to the total amount of fees paid to the havven holders in the following period, so that \(F_{r,t}
= F_{t-1}\). Substituting our earlier definition~\eqref{eq:feesreceived} for
\(\phi_{r,i}\) and solving for \(\phi_{base} \): \\

\begin{equation}
\phi_{base,t} \ = \ \frac{F_{t-1}}{\sum_{i} H_{i,t} \cdot \mathit{\Gamma}_{i,t}} \label{eq:feebase}
\end{equation} \\

\noindent We have now defined the maximum fee rate, \(\phi_{base}\), in terms
of the fees collected, \(F\). This rate should be achieved when an individual's
\(C_i\) was at \(C_{opt}\) in the previous period. \\

\noindent The definition of \(C_{opt}\) therefore provides the following
incentive. If \(P_n > \$1\) then the system encourages more nomins to be
issued. However, if \(P_n < \$1\), the system encourages nomins to be
burned.
