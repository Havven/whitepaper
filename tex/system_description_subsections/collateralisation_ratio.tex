\subsection{Collateralisation Ratio}
\label{subsec:collatratio}
\subsubsection{Optimal Collateralisation Ratio}

\noindent The optimal collateralisation ratio \(C_{opt}\) is a target for
havven holders to reach which brings the global \(C\) closer to \(C^*_{P_h}\).
If havven holders target \(C_{opt}\) effectively, the amount of fees they receive
is maximised.

\noindent \(C_{opt}\) is defined as a positive function of \(P_n\), such that its value
directly tracks changes in the nomin price; a havven holder wishing to
maximise their fees will target \(C_{opt}\) by issuing or destroying nomins
to manipulate their individual \(C_i\). \\

\noindent The function for \(C_{opt}\) given below provides our dynamic target
for havven holders based on the price of nomins:

\begin{gather} \label{eq:optcollateralisation}
\begin{align}
\begin{split}
C_{opt} \ &= \ f(P_{n}) \cdot C  \\ 
f(P_n) \ &= \ \max(\sigma \cdot {(P_n - 1)}^{\psi} + 1, 0) \\
\sigma \ & \text{ \ price sensitivity parameter } (\sigma > 0)\\
\psi   \ & \text{ \ flattening parameter } (\psi \in \mathbb{N} \text{, } \psi \nmid 2) \\
\end{split}
\end{align}
\end{gather}


\begin{center}
\begin{tikzpicture}[scale=1.15]
\begin{axis}[
    axis lines = left,
    xlabel = \(P_n\),
    ylabel = {\(f(P_n)\)},
    xtick = {1},
    ytick = {1},
    y label style={at={(axis description cs:.15,1.00)}, rotate=270,anchor=south},
    x label style={at={(axis description cs:1.05,.13)},anchor=north},
]
\addplot[domain=0:2, range=0:2]{(x-1)^3 + 1};
\addplot[domain=0:2, range=0:2, dashed]{x};
\addplot[dashed, samples=50, smooth,domain=0:6] coordinates {(1,1)(1,0)};
\end{axis}
\end{tikzpicture}
\end{center}

\noindent When \(P_n\) is at \$1, no signal is available to suggest that
the global nomin supply is incorrect.
In this case \(C_{opt} = C\) and no incentive exists to move away
from current collateralisation levels.
However, if \(P_n < \$1\), nomin supply outstrips demand; then \(C_{opt} < C\),
which encourages issuers to burn nomins, thereby raising the price. 
This movement of \(C_{opt}\) reflects the fact that \(C^*_{P_h} < C\)
under these circumstances. The \(P_n > \$1\) case is symmetric. \\

\noindent Notice that when the \(P_n\) is close to \$1, \( f'(P_n) \) is small,
so that Havven does not overshoot the correct nomin supply. However, the further it
diverges from \$1, the larger the slope becomes, providing a stronger fee
incentive for a havven holder to move toward
\(C_{opt}\). \\


\subsubsection{Maximum Collateralisation Ratio}

\noindent Havven seeks to maintain \(C \leq C_{opt} < C_{max} < 1\), in order
to retain sufficient havven collateral. It might seem intuitive that
\(C_{max}\) should be a static value. However, since \(C_{opt}\) varies linearly
with \(P_n\) and inversely with \(P_h\), there are situations where \(C_{max}\) may
need to change. We allow that a user may wish to set their \(C_i > C_{max}\), but
constrain by how much they may do so.

\noindent Below we define \(C_{max}\) in terms of \(C_{opt}\).

\begin{gather} \label{eq:maxcollateralisation}
\begin{align}
\begin{split}
C_{max} \ &= \ a \cdot C_{opt} \\ 
a & \geq 1 \\
\end{split}
\end{align}
\end{gather}

%\begin{center}
%\begin{tikzpicture}[scale=1.15]
%\begin{axis}[
%    axis lines = left,
%    xlabel = \(C_{opt}\),
%    ylabel = \(C_{max}\),
%    xtick = {1},
%    ytick = {1},
%    xticklabels = {,,},
%    yticklabels = {,,},
%    xtick style = {draw=none},
%    ytick style = {draw=none},
%    y label style={at={(axis description cs:0.18,1.00)}, rotate=270, anchor=south},
%    x label style={at={(axis description cs:1.08,0.14)}, anchor=north},
%]
%\addplot[domain=0:0.8]{1.25 * x};
%\addplot[domain=0:1, dashed, smooth]{x};
%\end{axis}
%\end{tikzpicture}
%\end{center}

\vspace{2mm}

\noindent The value of \(C_{max}\) determines how much collateral each nomin
is issued against. The higher it is, the more nomins can be generated for
the same value of havvens. By contrast, if \(C_{max}\) is low, the system
has a greater capacity to absorb negative shocks in the havven price before
it becomes undercollateralised. The value of \(C_{max}\) therefore represents a
tradeoff between \textit{efficiency} and \textit{resilience}. By defining
\(C_{max}\) as a function of \(C_{opt}\), Havven finds the optimal balance in
this dilemma. This ensures that like \(C_{opt}\), \(C_{max}\) only changes as a
consequence of instability in the nomin price. \\

\noindent It should be noted that \(C_{max} > 1\) corresponds to a fractional
reserve monetary system, where a greater value of nomins can be issued
against each havven. In Havven, this situation is unsustainable because it
would cause simultaneous appreciation of havvens (up to at least the value of
nomins issuable against a havven) and depreciation of nomins, immediately
diminishing \(C\), \(C_{opt}\) and \(C_{max}\), bringing them back under \(1\).
