\newpage
\subsection{Collateralisation Ratio}
\subsubsection{Optimal Collateralisation Ratio}

\noindent The optimal collateralisation ratio $C_{opt}$ is a target for havven
holders to reach in order to maximise the amount of fees they receive.
$C_{opt}$ is defined in terms of the nomin price $P_n$, such that its value directly tracks changes in the nomin price; a havven
holder wishing to maximise their fees will target $C_{opt}$ by issuing or destroying nomins. \\

\noindent The function for $C_{opt}$ given below provides our dynamic target
for havven holders based on the price of nomins:

\begin{gather} \label{eq:optcollateralisation}
\begin{align}
\begin{split}
C_{opt} \ &= \ f(P_{n}) \cdot C  \\ 
f(P_n) \ &= \ max(\sigma \cdot (P_n - 1)^{\phi} + 1, 0) \\
\sigma \ & \text{ \ price sensitivity parameter } (\sigma > 0)\\
\phi   \ & \text{ \ flattening parameter } (\phi \in \mathbb{N} \text{, } \phi \nmid 2) \\
\end{split}
\end{align}
\end{gather}


\begin{center}
\begin{tikzpicture}[scale=1.15]
\begin{axis}[
    axis lines = left,
    xlabel = $P_n$,
    ylabel = {$f(P_n)$},
    xtick = {1},
    ytick = {1},
    y label style={at={(axis description cs:.15,1.00)}, rotate=270,anchor=south},
    x label style={at={(axis description cs:1.05,.13)},anchor=north},
]
\addplot[domain=0:2, range=0:2]{(x-1)^3 + 1};
\addplot[domain=0:2, range=0:2, dashed]{x};
\addplot[dashed, samples=50, smooth,domain=0:6] coordinates {(1,1)(1,0)};
\end{axis}
\end{tikzpicture}
\end{center}

\noindent When $P_n$ is at \$1, $C_{opt} = C$ and there is no incentive given to move away from the current global collateralisation level. However, if $P_n < \$1$, then $C_{opt} < C$, incentivising issuers to burn nomins, thereby raising the price. The $P_n > \$1$ case is symmetric. Notice that when the $P_n$ is close to \$1, $ f'(P_n) $ is small. However, the further it diverges from \$1, the larger the slope becomes, providing a stronger incentive, in the form of potential fees, for a havven holder to move toward $C_{opt}$. \\

\newpage

\subsubsection{Maximum Collateralisation Ratio}

\noindent Havven seeks to maintain $C \leq C_{opt} < C_{max} < 1$, in order to retain sufficient overcollateralisation. It might seem intuitive that $C_{max}$ should be a static value. However, since $C_{opt}$ varies linearly with $P_n$ and inversely with $P_h$,  there are situations where $C_{max}$ may need to change. Below we define $C_{max}$ in terms of $C_{opt}$. \\

\begin{gather} \label{eq:maxcollateralisation}
\begin{align}
\begin{split}
C_{max} \ &= \ a \cdot C_{opt} \\ 
a & \geq 1 \\
\end{split}
\end{align}
\end{gather}

\begin{center}
\begin{tikzpicture}[scale=1.15]
\begin{axis}[
    axis lines = left,
    xlabel = $C_{opt}$,
    ylabel = $C_{max}$,
    xtick = {1},
    ytick = {1},
    xticklabels = {,,},
    yticklabels = {,,},
    xtick style = {draw=none},
    ytick style = {draw=none},
    y label style={at={(axis description cs:0.18,1.00)}, rotate=270, anchor=south},
    x label style={at={(axis description cs:1.08,0.14)}, anchor=north},
]
\addplot[domain=0:0.8]{1.25 * x};
\addplot[domain=0:1, dashed, smooth]{x};
\end{axis}
\end{tikzpicture}
\end{center}

\noindent The value of $C_{max}$ determines how overcollateralised each nomin is at issuance. The higher its value, the more nomins can be generated for the same quantity of havvens. In contrast, if $C_{max}$ is low, the system has a greater capacity to absorb negative shocks in the havven price before it becomes undercollateralised.  The value of $C_{max}$ therefore represents a tradeoff between \textit{efficiency} and \textit{resilience}. By defining $C_{max}$ as a function of $C_{opt}$, Havven finds the optimal balance in this dilemma. This ensures that like $C_{opt}$, $C_{max}$ only changes as a consequence of instability in the nomin price. \\

\noindent It should be noted that $C_{max} > 1$ corresponds to a fractional reserve monetary
system, where a greater value of nomins can be issued against each havven. In Havven, this
situation is unsustainable because it would cause simultaneous appreciation of havvens (up to at
least the value of nomins issuable against a havven) and depreciation of nomins, immediately
diminishing $C$, $C_{opt}$ and $C_{max}$, bringing them back under $1$.

