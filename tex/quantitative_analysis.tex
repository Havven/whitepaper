\section{Quantitative System Analysis}

Although we have strong confidence in it already, the more Havven's stabilisation mechanism
stands up to attack, the more confident we can be of its ultimate validity.
Therefore we take the view that falsification is important in validating our
proposals. \\

\noindent Ultimately this must done empirically, but it is important to stress test
Havven extensively in our our preliminary examination of its dynamics.
Therefore in our quantitative analysis we seek above all to identify its failure
modes, and also to characterise not just \textit{whether} Havven stabilises
nomin prices, but also by \textit{how much}.\\

\noindent Our present confidence in Havven stems from the three distinct approaches 
we take in modelling it:
\begin{itemize}
    \item[] \textbf{Analytical}
    
    Expressing our system in the language of game theory we seek to gain insight
    into price equilibria and perverse incentives with pen, paper, and decades of
    microeconomic theory.

    \item[] \textbf{Simulationist}

    We implement a broad range of strategies as AI agents, and examine how the
    market responds under different initial conditions, with different constituent
    populations, and in response to external shocks.

    \item[] \textbf{Empirical}

    Initial releases of Havven will be invaluable in checking our assumptions.
    Observation of real market behaviour will allow us to understand much better
    how the it responds in different situations, and therefore how to choose good values
    for system variables.
\end{itemize}

\subsection{Game-theoretic modelling}
% \todo[inline]{Discuss game theoretic modelling.}

\subsubsection{Actor Definitions}
\paragraph{Havven Holder}
\emph{An investor who owns \HAV{} tokens.} \\

\noindent In order to purchase \HAV{} the expected return has to be greater than that of alternative investments (opportunity cost). The expected return of \HAV{} comes from:
\begin{enumerate}
\item{The fees received on escrowed \HAV{}.}
\item{An increase in $P_h$.}
\item{Seigniorage (i.e., an increment in $P_h$ implies that the investor can issue more \NOM{}, which may eventually have a larger value than his original investment.}
\end{enumerate}

\noindent At any moment in time, the investor must decide:
\begin{enumerate}
\item{Whether or not to issue new \NOM{} (assuming $U_i < U_{max}$), or to burn some of them. All \NOM{} are issued/burnt at the prevailing market exchange rate, denominated in ETH.}
\item{Whether to sell some quantity of \HAV{} in the market at price $P^M_{h,t}$. Only \HAV{} which haven't been escrowed can be sold. Otherwise they must burn \NOM{} to release them. }
\end{enumerate}

\paragraph{Nomin User}
\emph{A person who uses the \NOM{} token.} \\
 
\noindent In order to purchase \NOM{}, it must provide the user more utility than USD, since both have the same consumption value in the market. This utility may come from the properties of crypto. \\

\noindent At any time, they must decide: 
\begin{enumerate}
\item{Whether to buy or to sell \NOM{} at $P_n$.}
\end{enumerate}

\paragraph{Agent-based modelling} It has been observed that analytic methods are often difficult to
apply in the complex and dynamic setting of a market.
One suggested solution to this problem is \textit{agent-based modelling}.
Under this paradigm, we proceed by first defining rational agent behaviour
and then simulating the interplay of those strategies over time.
We seek to develop a more effective
method of characterising market behaviour and equilibrium prices than pure analytic reasoning.~\cite{poggio2001agent}\\

\noindent Such simulations also provide an immediate means of measuring
quantities of interest. Simply by observing
the model, we can discover how varying input parameters
affect system outputs in an experimental fashion.
One important corollary is that this is a way of extracting reasonable
settings for system parameters (such as fee levels) that might be difficult
to reason about \textit{a priori}. These systems, reactive as they are,
also provide a method for testing proposed remedies for any identified failure
modes.

% \todo[inline]{Discuss that the game theory conclusions can be simulated}

\noindent In sum, then, the modelling seeks to answer the following, among other questions:

\begin{itemize}
    \item Does the system stabilise its nomin price?
    \item Under what conditions can stability fail?
    \item What are reasonable initial settings for fees and other parameters?
    \item What effect does the utilisation ratio have on havven/nomin price ratio?
    \item What is an effective utilisation ratio?
    \item What is the effect of a direct redemption regime?
    \item What are the expected returns for havven-holders?
\end{itemize}

% \todo[inline]{Fuller description of the technicals of the modelling.}

% \subsection{Modelling Havven}

% \paragraph{Environment}

% \paragraph{Agents}

% \paragraph{Optimal strategy mix}

\noindent Please visit \href{http://research.havven.io}{\texttt{research.havven.io}} for a pre-alpha version of our model.

\subsection{Fee/velocity/return computations}
% \todo[inline]{Fee/velocity/return computations}

\pagebreak
