\section{Road Map}

The Havven system will be released in phases, described in the following
sections. This phased rollout accelerates market penetration while
allowing features to be introduced and monitored incrementally. The intended approach  
is to decentralise Havven as it is built, which will be accompanied by a gradual
increase in the issuance ratio while the system develops.
This provides a balance of safety and functionality until Havven is fully
mature.

\paragraph{Release Schedule}
\renewcommand{\arraystretch}{1.5}
\setlength{\tabcolsep}{13pt}
\begin{centering}
    \begin{tabular}{|c|l|c|}
        \hline
        \textbf{Type} & \textbf{Iteration} & \textbf{Release (2018)} \\
        \hline
        \hline
        eUSD & Ether-Backed Nomins & Q1 \\
        \hline
        \hline
        nUSD & System A: Static Foundation Issuance & Q2 \\ \cline{2-3}
        & System B: Dynamic Market Issuance & Q3 \\ \cline{2-3}
        & System C: Multi-Currency Nomins& Q4 \\
        \hline
    \end{tabular} \\
\end{centering}
\vspace{0.5cm}

\subsection{Ether-Backed Nomins (\texttt{eUSD})}

\noindent The ether-backed nomin system implements \texttt{eUSD}, an interim stablecoin
which is to operate while the havven-backed system is being developed.
Price stability is maintained by a pool of ether which backs the circulating stablecoin
supply. In this system, \texttt{eUSD} can be purchased from and sold into a pool
for \$1 worth of ether. The Havven foundation provides at least \$2 of ether collateral
for each issued \texttt{eUSD}, which ensures that the entire supply is redeemable for its
face value even in the face of the ether price falling by up to two thirds. Fees are
collected on transactions, and these fees are collected by havven owners. \\

\noindent The \texttt{eUSD} system, being based on ether collateral, has a constrained
maximum supply. Therefore, in order to scale, the system must move towards its ultimate
issuance mechanism, where nomins are backed with havvens.

\pagebreak
\subsection{Havven-Backed Nomins (\texttt{nUSD})}

\paragraph{System A} Static Foundation Issuance \\

\noindent The System A version of \texttt{nUSD} allows issuance of nomins against
the value of havvens up to a static ratio, which is set by the Havven foundation.
\texttt{nUSD} will be created directly in the issuer's wallet, and fees paid
proportionally with the number of issued nomins per user. Given that it's necessary to
encourage liquidity, but not all the mechanisms outlined in this paper will be operating
yet, issuance will be by the foundation itself, and potentially other white-listed
addresses it trusts. In this way, the stability of the token is maintained by direct
market intervention by the foundation. \\

\noindent As System A fundamentally changes the issuance mechanics, \texttt{nUSD} is a distinct
token from \texttt{eUSD}, even though both target a price of \$1. At the time of the \texttt{nUSD}
launch, the \texttt{eUSD} contract will enter a liquidation period, when \texttt{eUSD}
can be sold back to the contract. At the conclusion of this period, the contract will be destroyed.
Future updates to the Havven system will not entail the destruction of any nomin tokens;
thus both \texttt{havvens} and \texttt{nUSD} will persist without further interruption. \\

\noindent System A limits issuance so that only those havvens the foundation controls can be
used to back nomins, which means the full value of the havven network cannot be deployed for
issuance. At this stage the market will be monitored closely by the Havven foundation, 
and the issuance ratio maintained at a low level or gradually ramped up. 
Yet this version of the system can't be opened up to the market at large, since not only are
the incentive mechanisms described in this paper not operational, but nomins are created
directly in the issuer's wallet. This may limit liquidity if issuers choose not to sell those
nomins, and is counterproductive to the general goal of stability. Systems B aims to combat
such limitations. \\


\paragraph{System B} Dynamic Market Issuance \\

\noindent System B composes the complete mechanism as described in the main body
of this paper. The incentives will be activated so that users earn fees in
accordance with how effectively they stabilise the nomin price. 
System B will enforce that issuance occurs through an issuance controller or
decentralised exchange to ensure that new liquidity is injected directly
into the market at \$1 per nomin. The Havven system must manage open
issue/burn orders while also continually updating the price at which they are
offered.

\noindent These elements being in place, market players are encouraged to behave correctly.
Thus, this phase opens up the opportunity for anyone to engaging in the issuance process.
With the intended mechanisms operational, any cap on the ultimate scale of the Havven system
is effectively removed, and the most important remaining task is stimulating demand for nomins. \\


\paragraph{System D} Multi-Currency Nomins \\

\noindent One means of increasing usage of nomins is offering it
denominated against a variety of different currencies.
In principle, the Havven mechanism can target the price of any asset.
With this in mind, the foundation intends to allow issuance
of different flavours of nomins. For example, in addition to 
nUSD, the system could also allow nKRW to be issued, tracking
the price of the Korean won. In such a system, issuers would be
rewarded with fees only for currencies they have issued.
This extension would replicate the Havven mechanism for each currency. 
If such issuance occurs against the same collateral pool, it is
straightforward to see that in the absence of other incentives,
the issued supply of each nomin flavour should be proportional with the
each flavour's share of total transaction volume. \\

\noindent A multicurrency nomin system clearly has the advantage of providing
the benefits of stablecoins to many markets, and thus of deepening
the fee-generating volume for havvens. But that these nomin flavours
are interrelated by the common pool of capital provides extra utility
in the form of relatively cheap foreign exchange in nomins, using 
havven issuance to settle between them.
The introduction of new nomin flavours will become straightforward,
requiring only the creation of a new oracle for the intended asset,
which does not have to be a currency. \\

Once this version of the system is operational, additional extensions to its
capabilities are anticipated for future work. \\


\subsection{Continuing Development}

\noindent The Havven protocol is not tied to any particular distributed ledger,
only requiring general smart contract capabilities and a modicum of speed.
However, its ultimate success is tied to the solution of a number of
fundamental problems in the blockchain space, and to the quality of its own
mechanisms. To this end, research and development efforts will continue in a
number of relevant areas, which may include:

\begin{itemize}
    \item Fast decentralised oracles.
    \item Alternative (scalable) blockchains and cross-chain functionality.
    \item Improved fee structures, including dynamic fees and hedging charges.
    \item Refined stabilisation mechanisms.
    \item System parameter optimisation.
    \item Stablecoin econometrics and modelling.
\end{itemize}

\noindent In addition to these fundamental questions, the Havven foundation will continue
to pursue integrations with projects which would benefit from a stable unit of
account. This encompasses any platform needing a settlement token which
integrates with decentralised systems. All such efforts aim to increase nomin
transaction volume, increasing the resources available for havven holders to
stabilise the token.
