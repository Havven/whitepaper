\section{System Analysis}

While the simplicity of the Havven mechanism makes it feel intuitively
viable, we take the view that falsification is vital in validating a proposed
cryptoeconomic system. The more resilient a given system is to hypothetical
attacks, the more trust can be put in its viability. \\

\noindent Ultimately this must done empirically, but it is also important to
model Havven extensively before launch. Therefore in our quantitative
analysis we seek above all to identify its failure modes, and also to
characterise its stability under a range of conditions.\\

\noindent In our quantitative analysis, we take three distinct approaches in modelling the system:
\begin{itemize}
    \item[] \textbf{Analytical}
    
	By expressing our system in the language of game theory and microeconomics,
	we seek to gain insight into Havven's incentive structure and the resulting
	price equilibria. Examining the problem from this direction can lead us to
	concise and mathematically robust conclusions.

    \item[] \textbf{Simulationist}

    We implement a broad range of strategies as AI agents, and examine how the
    market responds under different initial conditions, with different constituent
	populations, and in response to external shocks. This approach allows us to
	examine situations which are analytically intractable.

    \item[] \textbf{Empirical}

The initial release of the Havven system utilising ether-backed nomins will
be invaluable in testing our assumptions. Observation of real market
behaviour will allow us to better understand how the it responds in different
situations, and therefore how to choose appropriate values for system
variables. \end{itemize}

\noindent The results of these investigations will be published as they are completed.

\subsection{Agent-Based Modelling}
It has been observed that analytic methods are often difficult to
apply in the complex and dynamic setting of a market.
One suggested solution to this problem is \textit{agent-based modelling}.
Under this paradigm, we proceed by first defining rational agent behaviour
and then simulating the interplay of those strategies over time.
We seek to develop a more effective
method of characterising market behaviour and equilibrium prices than pure analytic reasoning. \\
%~\cite{poggio2001agent}\\

\noindent Such simulations also provide an immediate means of measuring
quantities of interest. Simply by observing
the model, we can discover how varying input parameters
affect system outputs in an experimental fashion.
One important corollary is that this is a way of extracting reasonable
settings for system parameters (such as fee levels) that might be difficult
to reason about \textit{a priori}. These systems, reactive as they are,
also provide a method for testing proposed remedies for any identified failure
modes, and are a platform to simulate the conclusions of any antecedent game-theoretic
reasoning. \\

\subsection{Expected Market Players}

Here we outline some of the players anticipated in the market. These
represent only some of the agents that our modelling and simulations are predicated upon.

\begin{itemize}
	\item[] \textbf{Havven Holders}

	A havven holder provides the collateral and liquidity for the system.
	It is assumed that havven holders primarily seek fee revenues, and escrow most of their havvens,
	adjusting their issuance to track Havven's moving fee incentives.
	While these incentives make sense if havvens are relatively stable in the long term,
	Havven will also provide incentives for correcting the nomin price in in the short term.
	Returns for these actors are primarily realised in fees, seignorage, and the appreciation of havvens resulting
	from their constrained supply.

	\item[] \textbf{Nomin Users}

	These are the market participants who will make up the base demand
	for any stablecoin, chasing its superior utility as a medium of exchange or as
	as means of hedging against other forms of value. 
	The users of nomins may include merchants, consumers, service providers, cryptocurrency market actors
	such as exchanges. 
	
	The transaction volume these users provide is necessary for fees to exist.
	They may be disincentivised from using the system in low liquidity situations or with excessive volatility
	in the price of nomins.

	\item[] \textbf{Speculators}
	
	Speculators may tend to magnify price corrections, and are a significant vector by which to introduce
	exogenous shocks to the system. In our modelling we induce volatility by simulating modes of interest
	such as large capital flows in response to breaking news and the like.

	Speculators also produce an important stabilisation force. When the market believes that the price is
	being stabilised, upward price shifts induce sell pressure, and downward
	price shifts induce buy pressure. This strategy is profitable on the assumption that the price
	will return to the equilibrium point. This neutral stabilisation force is a self-sustaining
	negative feedback loop which operates independently of other incentives; preliminary simulations and observations of other systems
	have verified the efficacy of this corrective pressure.
	
	%\item{Malicious Attackers}
	
	%We should examine what happens if a George Soros (or otherwise) attacks Havven.
	% \todo[inline]{List the various possible attacks against the system.}

	\item[] \textbf{Buyer of Last Resort}
	
	While the system is designed to work without intervention, the Havven foundation
	will have capital reserves with which to intervene in the market to stabilise
	nomin prices in extreme situations. 
	
	The advantage that such a market participant confers, given that a very large market entity is
	willing to underwrite the stability of the coin, is that profit strategies predicated upon the
	stability of the token become less risky and so more feasible. As a result, 
	any such presence, even if it rarely intervenes in the market, enhances the
	aforementioned neutral stabilisation force. In our modelling, the Havven foundation in this capacity takes on the role of providing confidence.
	
	Modelling results will inform the need of such an actor in the ecosystem.

	\item[] \textbf{Arbitrageurs and Market Makers}

	The arbitrage force allows us to assume that the havven/nomin, havven/fiat, nomin/fiat
	prices are properly in alignment or will soon become aligned. Market-making activities
	allow us in our modelling to neglect the bid/ask spread, and situations where there is insufficient
	liquidity for players to transact.
	
\end{itemize}

\noindent Please visit \href{http://research.havven.io}{\texttt{http://research.havven.io}} for an alpha version of our model,
and \href{http://blog.havven.io}{\texttt{http://blog.havven.io}} for further discussion of stablecoins and
cryptoeconomics.

\pagebreak

% \todo[inline]{Discuss that the game theory conclusions can be simulated}

% \noindent In sum, then, the modelling seeks to answer the following, among other questions:

% \begin{itemize}
%     \item Does the system stabilise its nomin price?
%     \item Under what conditions can stability fail?
%     \item What are reasonable initial settings for fees and other parameters?
%     \item What effect does the utilisation ratio have on havven/nomin price ratio?
%     \item What is an effective utilisation ratio?
%     \item What is the effect of a direct redemption regime?
%     \item What are the expected returns for havven-holders?
% \end{itemize}

% \todo[inline]{Fuller description of the technicals of the modelling.}

% \subsection{Modelling Havven}

% \paragraph{Environment}

% \paragraph{Agents}

% \paragraph{Optimal strategy mix}


% \subsection{Fee/velocity/return computations}
% \todo[inline]{Fee/velocity/return computations}

% \section{Qualitative System Analysis}

% This section provides a qualitative treatment of the reaction of the Havven system in response to various scenarios listed below:

% \todo[inline]{Scenario analysis.}

% \subsection{Incentives}

% \paragraph{Why would anyone use nomins?}
% \begin{itemize}
% 	\item Versus havvens?
% 	\item Versus alternatives?
% \end{itemize}

% \paragraph{Why would anyone issue nomins?}
% \begin{itemize}
% 	\item Fees
% 	\item Because they thought the peg would break in the positive direction.
% \end{itemize}

% \subsection{Scenarios}

% \subsubsection{Havvens appreciate against nomins.} \label{sec:havven-nomin-appreciation}
% \begin{itemize}
% 	\item{Havven-holders can issue more nomins.}

% 	This might be scary because we either want the nomin supply to fall, or nomin demand to increase.
% 	This is fine if the nomin price fell, because then nomin demand should increase.
% 	However, if it was simply that the havven price increased, then this might encourage
% 	oversupply of nomins as issuers compete for shares of the fee pool.

% 	\item{Cheaper exit}

% 	Anyone who has previously issued havvens but who wants to exit can buy nomins for
% 	cheaper than they issued them at, so egress with a profit.
% 	Alternatively, if the havven price doubles, then only half of their stake is required
% 	to back the nomins they have issued. They can use these proceeds to completely liquidate
% 	their position.

% 	\item{Each nomin locks fewer havvens.}
	
% \end{itemize}

% \subsubsection{Havvens depreciate against nomins.} \label{sec:havven-nomin-depreciation}

% \begin{itemize}
% 	\item{Havven-holders can issue fewer nomins.}
	
% 	Perhaps they may even be under-staked.

% 	\item{Each nomin locks more havvens.}
% 	\item{A player can issue a quantity of nomins and sell them for havvens.}
		  
% 	They would do this on the assumption that, once the nomin price decreases as a result of the increased
% 	supply, they will be able to buy back the same quantity of nomins to free up their
% 	havvens more cheaply.
% \end{itemize}

% \subsubsection{Nomin/havven liquidity dries up}
% \begin{itemize}
% 	\item Low nomin supply

% 	Nomins should appreciate against the havven. See \cref{sec:havven-nomin-depreciation}.
% 	Fee takings will decrease, hurting havven-holders in the long run,
% 	which should incentivise them to inject nomins into the ecosystem.

% 	\item Low havven supply

% 	If system backers are sitting on a pile of havvens, then the havven should
% 	appreciate. See \cref{sec:havven-nomin-appreciation,sec:havven-appreciation}.
% \end{itemize}


% \subsubsection{Long-run havven price appreciation}\label{sec:havven-appreciation}
% \todo[inline]{Long-run havven appreciation scenario.}
% In this instance, over the long run, the price of havvens could appreciate quite substantially,
% which will lead to an increase in nomin issuance rights (and so probably actual nomins in circulation).
% There are at least two reasons that havvens could appreciate:
% \begin{itemize}
% 	\item Increased nomin velocity/demand, leading to greater fees;
% 	This is fine: in this case, nomin issuance rights increase in proportion with demand for the currency.

% 	\item Speculation
% 	This is not so fine: in this case, the nomin supply can be expanded without any accompanying
% 	expansion in its demand, which will depress the nomin price.
	% \todo[inline]{Resolve disconnect between speculation-driven value of havvens and nomin supply.}
% \end{itemize}

% \subsubsection{Long-run havven price depreciation}\label{sec:havven-depreciation}
% \todo[inline]{Long-run havven depreciation scenario.}

% \subsubsection{Radical shifts in usage}
% \todo[inline]{Usage-shift scenario.}

% \todo[inline]{List expected players in the market.}
% \todo[inline]{Outline incentives and actions for different players.}

% \pagebreak