\section{Introduction}

\subsection{Money and Cryptocurrencies}

Money has become almost invisible over the past few decades as payment technology has advanced. The technology of money has three key functions: to act as a unit of account, a medium of exchange
and as a store of value. In addition, money should ideally exhibit durability, portability, divisibility, uniformity, limited supply, and acceptability. But it is often lost upon users of money that it is itself a technology that can be improved. Specifically, this means improving the performance of our six desirable properties. \\

\noindent Bitcoin is an impressive technological advancement on existing forms of money because it
simultaneously improves durability, portability, and divisibility.
Further, it does so without requiring centralised control or the enforcement of a nation state from which to derive its value. It is precisely its fixed monetary policy which has protected Bitcoin from debasement and
devaluation, allowing it to outperform other forms of money as a store of value, and increased adoption
has tended to drive the price up over time. Unfortunately, the fixed money supply has also created the
potential for short-run volatility as there is no mechanism within Bitcoin that can dynamically
adjust to changing demand. \\

\noindent Bitcoin has thus tended to be a poor medium of exchange and an even worse unit of account.
In order for something to perform well as a medium of exchange or unit of account it must remain
relatively stable against goods and services. If the price of money is too variable then it becomes less useful as a
denominator of other goods.

\subsection{Stablecoins}

\noindent A stablecoin is a cryptocurrency designed for price stability, such that it can function both as a medium of exchange and unit of account. It should ideally be as effective for making payments
as fiat currencies like the US Dollar, but still retain the desirable characteristics of Bitcoin, namely
transaction immutability, censorship resistance and decentralisation. \\

\noindent Cryptocurrencies are in these ways a far better form of money but have been significantly hindered
in their adoption by the volatility of the inflexible monetary policies of decentralised systems. Stability continues to be one of the most valuable and yet the most elusive
characteristics for the technology. Clearly, the ability to create alternative and dynamic monetary policies within
cryptoeconomic systems is still nascent, and significant research into stable monetary
frameworks for cryptocurrencies is required.

\subsection{Havven}

\noindent The Havven stablecoin system is a novel form of representative money in which there is no requirement for a physical
asset, thus removing problems of trust and custodianship. The asset used to back the stablecoin is 
a pool of reserve tokens that collectively represent the system itself. Controlling these reserve tokens reflects participation in the Havven system,
and are a proxy for its value. Havven generates fees from users who transact in the stablecoin and distributes them among the holders of the reserve token, compensating them for underpinning the system.
Havven therefore rewards those who actively participate in maintaining the stability of the system and charges those who benefit from its utility. These rewards are provided in response to the active management of the supply of the exchange token such that its price mirrors that of the asset it tracks. \\

\noindent Because we have created a system that generates cash flow for participants, we now have an asset which can be used as the collateral to support the stablecoin with a well-defined market value. The key to this is that the value of the
system is measured in USD. This allows the system to issue a stablecoin which can be presented and redeemed for a
percentage of the collateral tokens valued at 1 USD. Backing a stablecoin in this way is beneficial because
such a cryptoeconomic system does not require trust in a centralised party; each participant has full
transparency over how many tokens have been issued against the available collateral at all times. \\

\noindent The two linked tokens and the complex of incentives for stability are defined below:

\paragraph{Havvens:} The collateral token, whose supply is static.
The capitalisation of the havvens in the market reflects both the system's aggregate value and the reserve
which backs the stablecoin. Thus, users who hold havvens take on the role of maintaining stability.
Following bitcoin, the Havven system will appear in upper case and singular; while the havven token will be lower case and may be plural.

\paragraph{Nomins:} The stable exchange token - the stablecoin - whose supply floats. Its price measured in fiat currency should be relatively stable.
Other than price stability, the system should also encourage some adequate level of liquidity for nomins
to act as a useful medium of exchange. \\

\noindent Each holder of havvens is granted the right to issue a value of nomins in proportion to the USD value
of the havvens they hold and are willing to place into escrow. If the user wishes to release their escrowed havvens, they must
present the system with nomins in order to free their havvens and trade them again.
The holders of this token provide both collateral and liquidity, and in so doing assume some
level of risk. To compensate this risk, such nomin-issuers will be rewarded with fees the system levies
automatically as part of its normal operation. \\

\noindent This issuance mechanism allows nomins to act as a form of representative money, where 
each nomin represents a share in the havven value held in reserve. Nomins derive value insofar as they provide
a superior medium of exchange, and are effectively redeemable for a constant value
of the denominating asset. In this paper, we use USD as this asset, but this could be any external
and appropriately fungible asset, such as a commodity or a fiat currency.  \\

\noindent In this manner, the system incentivises the issuance and destruction of nomins so that the value of
the nomin pool expands and contracts in proportion with the total value of havvens backing them.
If prices change exogenously, then the system is designed to provide incentives for actors to
counteract that change. \\

\noindent The Havven system is relieved of the obligation to respond to major macroeconomic conditions, 
as it benefits from the stabilisation efforts of large institutions acting in fiat markets.
In addition, as Havven has the freedom to significantly overcollateralise its pool of circulating currency, it
insulates itself against dramatic corrections in the havven market.
Havven therefore acts as a bridge between fiat currency and cryptocurrency as a hybrid of two technologies and possessing
the advantages of both. \\

\noindent Clearly, the introduction of a new cryptocurrency in isolation offers no additional value given
the existing and established alternatives such as Bitcoin or Ethereum. Havven thus seeks to derive value
from the addition of \textbf{stability} to its inherited properties as a modern cryptocurrency.
It is designed to provide a practical medium of exchange, without compromising the benefits that
decentralisation offers in order to substantially improve the technology of money.
There are many applications which Bitcoin's inherently deflationary monetary policy and
volatility presently make impossible: any token which is able to demonstrate an increment
in utility over both fiat and cryptocurrencies in these dimensions  will significantly
enhance the uptake of cryptoeconomic technology globally. \\

\noindent Finally, the design choice to back the system with a self-referential token was obvious; an asset-backed stablecoin with a cryptocurrency basket as reserve will always be inherently volatile, despite diversification, and will never be able to achieve the bespoke functionality of an asset which derives its value from stability.

\pagebreak
