\section{Introduction}

\subsection{Payment Networks}

\noindent Payment networks are closed systems within which users can transfer
value. Such systems include credit card networks, the SWIFT network, and
PayPal. Proprietors of these networks possess absolute control over the value
within the network, so any transaction conducted within them may be blocked
or reversed at any time. Although this is ostensibly designed to protect
users, it introduces systemic risk for all participants. If the network is
compromised or its owners cease to behave benevolently, no party can trust
that the value in their account is secure or accessible. \\

\noindent In a traditional payment network like American Express,
participants trust that the fees charged are sufficient to service the
expenses incurred. However, were this trust to disappear, merchants would
refuse to participate. Thus, the value of the unit of account within this
network is derived solely from a single entity and the trust that
participants have in that entity. As a result, the viability of any
centralised payment network depends on complete trust in a central authority.
\\

\noindent Bitcoin solved these problems by ensuring that users have sole
discretion over the money in their account by producing a trustless,
permissionless payment network in which anyone could participate at will.
Since users could enter and exit the system at any time without being exposed
to the aforementioned risks, adoption was accelerated, and network effects
were amplified. Programmable blockchains allow the logic of a payment network
to be decentralised in a transparent way. Anyone can verify whether the
network is solvent, reducing the systemic risk associated with centralised
networks.

\subsection{Cryptocurrency}

\noindent The technology of money has three key functions: to act as a unit
of account, a medium of exchange and a store of value. As payment technology
has advanced in recent years, money has become increasingly invisible and it
is often lost upon its users that, like any technology, it can be improved.
Bitcoin and other cryptocurrencies represent an impressive technological
advancement on existing forms of money because they deliver improved
durability, portability, and divisibility. Further, they do so without
requiring centralised control or sovereign enforcement from which to derive
their value. Their fixed monetary policies have protected them from
debasement and devaluation, allowing them to outperform other forms of money
as a store of value. However, this has created the potential for short-run
volatility as they lack mechanisms to dynamically adjust supply to changing
demand. Bitcoin has thus tended to be a poor medium of exchange and an even
worse unit of account. In order for a token to perform these functions its
purchasing power must remain relatively stable against the price of goods and
services over the short to medium term.

\subsection{Stablecoins}

\noindent Cryptocurrencies exhibit transaction immutability and censorship
resistance, and in these ways are a better form of money; but their adoption
has been hindered by the volatility inherent in their static monetary
policies. Users cannot engage with such systems as a medium of exchange if
the purchasing power fluctuates. Stability continues to be one of the most
valuable yet elusive characteristics for the technology. \\

\noindent Stablecoins are cryptocurrencies designed for price stability. They
should ideally be as effective at making payments as fiat currencies like the
US Dollar, while retaining their other desirable properties. A decentralised
payment network built on a stablecoin would be able to capture all the
benefits of a permissionless system, while also eliminating volatility. One
approach to achieving price stability is to produce a token whose price
targets the value of a fiat currency. Targeting stability against fiat
currencies obviates the need to respond to macroeconomic conditions, as the
token then benefits from the stabilisation efforts of large institutions
acting in fiat markets. Furthermore, if a token’s price can be maintained at
\$1, then it can serve as an interface between fiat money and cryptocurrency.
If such a stablecoin does not require an account in a traditional bank, then
it can be effectively used for settlement and purchasing, without the
centralisation and counterparty risk involved in fiat transactions. Thus it
can be expected that by using stablecoins, exchanges that trade fiat for
crypto will be able to rapidly reduce their transactional costs, reducing the
barriers for new users to enter the market.

\subsection{Distributed Collateral}

\noindent Today’s fiat money is not backed by an asset; its stability is
derived from the authority of the governments which issue it. These
governments require that tax obligations are denominated in the currencies
they control, which are then used to fund active stabilisation efforts.
However, with government control comes the risk of tyranny and debasement.
Decentralised monetary systems don’t have these powers, and so they must use
collateral to provide confidence in the value of their tokens. A
decentralised system cannot use collateral assets that exist outside the
blockchain, as interfacing with these assets necessitates centralisation with
the aforementioned failure modes. Meanwhile, cryptoasset prices have been
dominated by speculative volatility. So whether a system uses real-world
assets or cryptoassets to back a stable token, if the value of the collateral
is uncorrelated with the demand for the token, then the system is vulnerable
to external price shocks. Large corrections can destroy the value of
collateral without any change in the demand for the token issued against it.
Clearly then, in designing an asset-backed stablecoin it is important to
select the collateral asset carefully, but no existing asset perfectly serves
the purpose.

\subsection{Havven}

\noindent Havven is a decentralised payment network where users transact
directly in a price-stable cryptocurrency. Those who use the stablecoin pay
fees to those who collateralise the network, compensating them for the risks
of providing collateral and stability. Collateral providers control the money
supply, and fees are distributed in proportion with each individual’s
stabilisation performance. Thus, Havven rewards suppliers of stability and
charges those who demand it. \\

\noindent Havven implements two linked tokens to achieve this structure: \\

\noindent \textbf{Nomin}

\vspace{1mm}

\noindent The stablecoin, whose supply floats. Its price as measured in fiat
currency should be stable. This token is useful insofar as it provides a
superior medium of exchange. Thus in addition to price stability, Havven
should encourage adequate nomin liquidity.

\vspace{2mm}

\noindent \textbf{Havven}

\vspace{1mm}

\noindent This token provides the collateral for the system and has a static
supply. Its market capitalisation reflects the system’s aggregate value.
Ownership of havvens grants the right to issue a value of nomins proportional
to the dollar value of havvens placed into escrow. If a user wishes to
release their escrowed havvens, they must first present the system with the
quantity of nomins previously issued \footnote{Following Bitcoin, the Havven
system will appear in uppercase and singular; while the havven token will be
lowercase and may be plural.}. \\

\noindent The havven token is a novel decentralised asset, whose intrinsic
value is derived from the fees generated in the network it collateralises.
This enables a form of representative money in which there is no requirement
for a physical asset, thus removing the problems of trust and custodianship.
Issuance of nomins requires a greater value of havvens to be escrowed in the
system, providing confidence that nomins can be redeemed for their face value
even if the price of havvens falls. The system incentivises the issuance and
destruction of nomins in response to changes in demand, but ultimately the
intrinsic value of the havvens will reflect the required nomin supply.
Backing a stablecoin in this way provides full transparency over how many
tokens have been issued against the available collateral. This provides a
solid basis for confidence in the solvency of the payment network built upon
it.\\

\noindent Denominating the value of the nomin in an external fiat currency
means that stability is relative only to that currency. Initially this
currency will be the US dollar, but in the future the system may support
additional flavours of stablecoin that are denominated in other currencies.
The interested reader can find additional discussion of payment networks,
pstablecoins, and cryptoeconomics at http://blog.havven.io.\pagebreak
