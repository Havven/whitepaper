
\section{System variables}

\todo[inline]{More complete system variable section.}

\noindent What follows are the main variables of the system. Under each heading, each row will correspond to a single quantity of interest. Each row will have three columns. Leftmost, a mathematical definition of the variable; in the middle, the dimension of the quantity (which units it is measured in); and on the rightmost, a short English summary of the variable.\\

\noindent Certain abbreviations will be used. For example, \(\HAV{}\) and \(\NOM{}\) will be used as abbreviations for havvens and nomins considered as units of measurement. \\

\paragraph{Prices}
\begin{align*}
    P_h & \ && &(\frac{\text{\$}}{\HAV{}}) && &\text{: havven price.} \\
    P_n & \ && &(\frac{\text{\$}}{\NOM{}}) && &\text{: nomin price.} \\
    \pi &:= \frac{P_h}{P_n} \ && &(\frac{\NOM{}}{\HAV{}}) && &\text{: havven to nomin conversion factor.} \\
    P_h' &= f(V_n, V_h) \cdot R && &(\frac{\text{\$}}{\NOM{} \cdot \text{sec}}) && &\text{: havven price rate of change.}
    \intertext{Here \(R\) is a risk term incorporating, for example, volatility, number of buyers versus sellers, and so on.}
\end{align*}
\\


\paragraph{Money Supply}
\begin{align*}
    &H \ && &(\HAV{}) && &\text{: Quantity of havvens, which is constant.} \\
    &H_e \ && &(\HAV{}) && &\text{: Quantity of escrowed havvens.} \\
    &N = H_N \cdot \pi \ && &(\NOM{}) && &\text{: Quantity of nomins. This can float.} \\
    &H_N = \frac{N}{\pi} \ && &(\HAV{}) && &\text{: Havven value of issued nomins.}
    \intertext{Ideally, \(H_N \leq H_e\).}
\end{align*}
\\

\paragraph{Utilisation Ratios}
\begin{align*}
    &U = \frac{H_N}{H} \ && &\text{(dimensionless)} && &\text{: Empirical issuance ratio. } \\
    &U_{max} \ && &\text{(dimensionless)} && &\text{: Targeted issuance ratio ceiling.}
    \intertext{\(0 \leq U \leq U_{max} \leq 1\)}
\end{align*}
\\

\paragraph{Microeconomic Variables} These should be defined as functions of \(P_n, \ P_v, \ \text{fees, etc.}\)
\begin{align*}
S_n \ && (\frac{1}{\text{sec}}) && &\text{: average nomin spend rate} \\
S_i \ && (\frac{1}{\text{sec}}) && &\text{: average issuance rate} \\
S_r \ && (\frac{1}{\text{sec}}) && &\text{: average redemption rate}
\end{align*}
\\

\paragraph{Money Movement}
\begin{align*}
    V_n &= S_n \cdot N \ && &(\frac{\NOM{}}{\text{sec}}) && &\text{: nomin transfer rate.} \\
    V_v &= V_i + V_r \ && &(\frac{\HAV{}}{\text{sec}}) && &\text{: nomin} \leftrightarrow \text{havven conversion rate.} \\
    V_i &= (C - C_N) \cdot S_i \ && &(\frac{\HAV{}}{\text{sec}}) && &\text{: nomin issuance rate.} \\
    V_r &= C_N \cdot S_r \ && &(\frac{\HAV{}}{\text{sec}}) && &\text{: havven redemption rate.} \\
    \intertext{\(V_i\) is assumed to grow as there are more free havvens in the system.}
    \intertext{\(V_r\), by contrast, is taken to grow proportionally with the number of escrowed havvens.}
\end{align*}
\\

\paragraph{Fees}
\begin{align*}
\intertext{The following fees are ratios, for example 0.1\%, levied on each transaction.}
&F_{nx} & \ && &\text{(dimensionless)} && &\text{: nomin transfer fee} \\
&F_{cx} & \ && &\text{(dimensionless)} && &\text{: havven transfer fee} \\
&F_i & \ && &\text{(dimensionless)} && &\text{: nomin issuance fee} \\
&F_r & \ && &\text{(dimensionless)} && &\text{: havven redemption fee} \\
% \intertext{These quantities are the aggregated fees accrued by the system per unit time.}
% &Ag_{nx} &:= V_n \cdot F_{nx} \ && &(\frac{\NOM{}}{\text{sec}}) && &\text{: fees taken from nomin transfers.}
\end{align*}

\pagebreak
