\documentclass{article}
\usepackage[utf8]{inputenc}
\usepackage{amssymb}
\usepackage{mathtools}
\usepackage{svg}
\usepackage{cite}


\usepackage{draftwatermark}
\SetWatermarkText{DRAFT}
\SetWatermarkScale{3}

\begin{document}

% Macros
\newcommand{\CUR}{\textsc{cur}}
\newcommand{\NOM}{\textsc{nom}}


\title{Havven - A Stablecoin \\ v0.1}
\author{Anton Jurisevic, Samuel Brooks, Kain Warwick}
\date{October 2017}


\begin{figure}
    \centering
%    \includesvg[width=0.33\textwidth]{block8logo}
\end{figure}
\maketitle

\section{Introduction}

Introduction. \\

\subsection{What is money?}

They key requirement of money is for it to hold a known value over an arbitrary period of time, such that rational actors are able to make economic plans. \\

\noindent Under the gold standard system of money, it was found that in times of economic crises this method of representing value caused exacerbations in poor economic conditions due to the tendency of rational actors to cut spending and 'hoard' cash. \\

\noindent Fiat currency was introduced to relieve this problem. Fiat has no intrinsic value or representative value, instead, it derives its (community market) value from the trust that is placed in the government which issues it. That is, fiat money is only worth what people think it's worth. This is just the same as with Bitcoin, however with Fiat, there is a central authority that has some globally known reputation, rather than a program (the Bitcoin protocol). Fiat then is not 'backed', but it is underpinned by the reputation of the issuing agency.

\subsection{Why stablecoins}

\noindent For cryptocurrencies, stability continues to be the most valuable and yet the most elusive characteristic. \\

\noindent Outline the current issues around why stablecoin (particularly the centralisation risks that Kain called out). \\

Use cases

- Kain to add.

\noindent While cryptocurrencies coexist with traditional fiat, there will continue to be an economic need for Havven's stable peg to fiat currencies. \\

\subsection{Havven, a stablecoin}

\noindent We would like to design a stablecoin; any currency which seeks to be a viable medium of exchange must have a sufficiently stable price relative to some acceptable denomination. \\

\noindent  In his discussion of Hayek money, Ametrano correctly makes the point that Bitcoin serves the purpose of crypto-gold much better than it does crypto-unit-of-account, due to its volatility and constrained supply.~\cite{ametrano2016hayek} By contrast, governments, who mint their own currencies, can and do execute discretionary stabilisation policies to manipulate the circulating supply. This kind of powerful lever is not available to Bitcoin and other supply-constrained currencies of its type, but a similar system whose monetary policy is algorithmically countercyclical rather than deflationary could inherit the desirable characteristics of both. \\

\noindent It should be possible however, by automatic means, to incentivise the issuance and destruction of tokens according to demand. In this way, users of this such a currency would be allowed to capitalise it while the system automatically seeks to expand and contract the money supply as its backing reserve fluctuates in value. By this mechanism we might produce a more perfect currency where supply floats with necessity, but which is not prone to debasement by selfish or misguided actors, or other issues commonly associated with deflationary forms of money. We should also seek to ideally remove some of the distortions created by traditional monetary policy, which, when it is expansionary, shrinks the purchasing power in every account which is not a direct beneficiary of that policy. \\

\noindent Hence, we propose a system of two intertwined currencies which serve very different purposes as a part of the Havven stablecoin system:

\paragraph{Curit} The reserve token. Users buy Curits to obtain a part share in the entire system. The holders of this token are providing collateral for the system, and in so doing, assuming some level of risk. To compensate this risk, Curit-holders will be rewarded with fees the system levies automatically as part of its normal operation. The capitalisation of the reserve market reflects the system's aggregate value.

\paragraph{Nomin} The exchange token. Philosophically, we would like the Nomin to be a utilon, a constant unit of utility, and so the system should stabilise its price in terms of some external, relatively-stable currency or commodity basket, in order to insulate the Nomin price from the volatility of its 
underpinning reserve. Each holder of Curits is granted the right to issue their own Nomins, in proportion to the value of the Curits they hold and are willing to escrow. If the user wishes to redeem their escrowed Curits, they must present the system with Nomins, in order to freely trade them again. Other than just price stability, the system should also encourage some adequate level of liquidity for Nomins to act as a useful medium of exchange. \\

\noindent Clearly, the introduction of a new cryptocurrency, in isolation, offers no additional value given the existing and established alternatives, such as Bitcoin and Ether. Havven then seeks to derive value from the addition of \textbf{stability} to its inherited properties as a modern cryptocurrency. \\

\noindent The Havven stablecoin system is akin to representative money in the sense that the fungible "Nomin" tokens of the Havven system represent some value held in reserve. We define the Curit to be the token of backing value as this is both the start and end point of using the Havven mechanism; Curits develop intrinsic value given their ability to develop stability (through Nomins) with an external denomination. Hence, Nomins have no intrinsic value as we define Curits as carrying the functional value associated with being able to derive a stable medium of exchange. It's representative money because Curits represent the tokenised ownership of the system that supports the functionality. If this relies on some implicit assumption that a group of actors think it has value, this is also valuable, given the friction/stickiness associated with an actor changing out their value store to some other form of money. \\

\noindent However, Havven is not representative money as we have traditionally known it. Historical instantiations, such as the gold standard which allowed anyone to claim against the reserve, caused exacerbations in times of economic turmoil. Havven instead provides an ability for Nomins to be pegged against some external form of value by regulating supply with clever market incentives.

\noindent The Havven system is designed such that the stable currency that is issued (Nomins) is both denominated and pegged against an external store of value. Throughout this paper we use USD as a the reference example, however this could be any external and appropriately fungible asset, such as a commodity or state currency (note, denominations in other cryptocurrencies are not necessary as these already benefit from the features Havven is implementing for the external denominator). \\

\noindent Nomins in the early stages of the system are anticipated to be used as a means to hedge against cryptocurrency volatility and as a settlement layer. For example, centralised cryptocurrency exchanges may use the Havven system to settle between themselves, expressing the value of the settled funds/assets in USD (for Nomins denominated in USD). Later, with greater maturity and scalability of Ethereum, Havven may be extended for more general purpose use. \\

\noindent Philosophically, we believe that Havven is a means to an end, and that is the accelerated and wider adoption of cryptoeconomic technology. \\

\noindent More information can be found on havven.io.


\pagebreak
\section{Functional description}

Havven works by providing a set of market incentives that support a stable peg.

\subsection{Stability design considerations}

Fundamentally, we wish to configure a system of cryptoeconomic incentives such that the value of the exchange token converges to 1 USD. This paper focuses on this value stabilisation as it is the key enabler to achieve a better form of money. \\

\noindent Let us consider the various ways in which one can maintain a stable value relative to a fiat currency. The question we wish to answer is, "How can we \textbf{influence the value} of a cryptocurrency such that one unit of the pegged currency matches one unit of the denominating currency?" This is a challenging scenario because there are multiple related forces at work on the value of both. We consider these as two independent groups: "market forces" and "control forces". \\

\noindent \textbf{Market} forces represent supply and demand. These are necessarily different for both currencies, otherwise the value of both currencies would move in unison. \\

\noindent \textbf{Control} forces then are the controls one is able to apply over a currency to control its value, such as an inflation rate or a buy-back scheme. \\

\noindent Our peg then should seek to tune the control forces such that one unit of a control currency equals one unit of the denominating currency. We assume that the forces for one currency are independent of the forces for another. \\

\noindent So what are the mechanisms we can apply to control the value of a currency? We consider:

\begin{itemize}
	\item Issuing new currency to increase supply (inflation)
	\item Buying back existing currency to decrease supply (deflation)
	\item Unilateral balance control (changing account balances to maintain a stable buying power)
	\item \textit{Others?}
\end{itemize}

\noindent Unilateral balance control, such as that as described in Amentrano's paper on Hyak Money, is discounted on the basis that individual balances being directly modified would be unpalatable to the general population. This leaves us with simply the forces relating to modifying supply. \\

\noindent We review the key incentive drivers in the design of an economically stable cryptocurrency. \\

\begin{itemize}
	\item Fees
	\item Supply control
	\item Capital growth
	\item Bonds
\end{itemize}

\noindent We consider the following significant questions:

\begin{itemize}
    \item How should Nomins be created and destroyed?
    \item How do we incentivise actors to contribute to system liquidity?
    \item Should fees only be given to those who have actually issued Nomins?
    \item Are transfer fees charged in only Nomins? Would actors not then just try to convert to Curits and exchange there?
    \item How do we select a utilisation ratio? What is its curve?
    \item \textit{@anton What is this one?} Utilisation ratio needs to have distinct parts. Number of escrowed Curits is not quite the same as the number of issued nomins.
    \item Should we allow the system to maintain a pool of Curits/Nomins that it itself can buy/sell?
    \item What if the system didn't burn the Nomins that were handed back when Curits were redeemed?
    \item Should Nomins, or Curits, be serialised?
    \item Should newly created Nomins go to escrowed Curit accounts, or be sold?
\end{itemize}

\subsection{Description of mechanism}

Curit holders have an ongoing option to commit to an escrow of their collateral token in return for an amount of Nomins valued at some fraction of the value of the Curits, denominated in USD. We call this fraction the Utilisation Ratio, (or later more formally, the Issuance Ratio), $ U $. \\

\noindent The Curit holder can then sell their Nomins for ETH for any price (or we can force a sale at the correct ETH price), and retain the ETH.


\subsection{Alternative approaches}

\noindent An autonomous system which mimics the actions of a central-bank, wherein reserves of cryptocurrency are held for the purposes of currency buy-back was also considered. This approach involves a complex system of managing bonds and Nomin issuance as coupons to bond-holders (similar to Basecoin), whilst also acting as a buyer of last resort for all Nomins sold under the peg beyond some threshold. This approach was discounted due to the need to anticipate complex market counter-algorithms to support the peg. \\

\noindent The Havven mechanism is a novel and simpler approach which uses only open market incentives for economically rational participants to bring stability to the exchange token. \\

\textit{(SGB: escrowing Curits and selling them for nomins by a Curit holder creates demand for when the holder needs to buy-back the nomins to get his Curits back. If we don't do this and hold the ether in reserve, we can automatically buy-back nomins much more efficiently. This is probably far more stable than having individuals control and possible holding onto the nomin supply, thereby creating large spikes in both demand and supply; a spike in demand could be felt when there's a run on Nomins to release Curits, this could be compensated for if other efficient market actors then escrowed their Curits to sell nomins at the higher price, but in the case that there are no more nomins to issue (all Curits are escrowed), the price of Nomins could skyrocket.)}

\subsection{Price discovery}

One of the key challenges with denominating a cryptocurrency in a fiat currency is the fundamental link this creates to the centralised world; when the denominating currency exists external to the blockchain ecosystem, some bridge must be built so that the system can act with knowledge of the outside world. Often, this is done by sacrificing trust; in order to reclaim system performance, we can trade some of the trustlessness of the design, such as through implementing an trusted "Oracle" service in order to gain knowledge of the external world and build a causal link. \cite{brooks2017blog}.\\

\noindent Havven achieves this causal bridge to USD through market arbitrage; on the Havven decentralised exchange (hosted on Ethereum), both Curits and Nomins are denominated in ETH. Ether, in turn, is assumed to have some instantaneous value in USD on one or more external markets. In this way, we can avoid a trusted bridge and instead reduce our price-finding requirement to the minimum viable assumption that there will always be some external market for ETH in USD (or some other chosen denomination) and that market actors will seek profit through arbitrage. \\

\noindent This method is superior to using an Oracle service pushing the exchange rate of ETH/USD at a regular cadence into the Havven decentralised exchange as it completely removes any risk associated with centralisation risks of such a process. \\

\subsection{Backing collateral}

Central to the design of the Havven money system is how the currency is backed by collateral. \\

\noindent For monetary systems that are backed by an external asset, centralisation risks are frequently encountered and are often without solution. The question arises: \\

\noindent How can one take an external asset and make it distributed such that we can mitigate the centralisation risk? \\

\noindent Finally, we arrive at the simplest and most ideal solution, being to use the system itself as the backing collateral. We can then issue an exchange token against this in a manner similar to fiat in that the exchange token has no intrinsic value, but similar to representative money as we argue that the collateralised value exists in the Curit token. \\

\noindent \textit{SGB: Critical issue: fiat works because you cannot copy a country. Havven may not work because you can easily copy the source code and start your own. So there needs to be a great deal of effort expended to gain market traction. This is where the ICO comes in and an open-ended one could work the best.} \\

\noindent The basis for Havven is an initial acceptance of the idea that the asset can be the system. The only other alternative is to use a basket of cryptocurrencies to back the exchange token, however this ultimately suffers from the same issues we are trying to prevent, namely volatility that is only mitigated through diversification of the crypto-assets chosen. \\

\noindent Finally, in backing by a real-world (external) asset, such as gold or a state currency, we encounter centralisation risks once again, including an inability to prove the existence of the backing asset (fraud risk). \\

\noindent Fundamentally, whenever you want to denominate a stablecoin in something from the real world (such as USD), then you will always need a bridge between the 'real-world' and the walled garden of the blockchain. \\

\noindent To this end, our design incorporates a special exchange (bridge) between fiat and Ether. This exchange is run as a not-for-profit by the Havven benevolent dictatorship. The price of Ether is then taken from this exchange and injected into the DEX (again, assuming scalability issues aside). The DEX then has an authoritive Ether/USD value  \\

\noindent => what happens when this exchange is shutdown by the feds? The whole system collapses. You will always need some kind of trusted source of information for the exchange rate if the denominating asset is outside the system. \\

\noindent \textit{SGB: Volume of reserves is also problematic if one is just using the backing system as collateral - this collateral pool (i.e. value of the functionality of maintaining stability plus the value of the generated fees?) is small relative to the value of Nomins on issue (i.e. what happens if the price of nomins goes up but the value of Curits stays the same?)} \\

\paragraph{Raised funds}

What happens with accumulated ICO funds beyond what is required to build and market the platform? Currently these do not "back" the platform; if the value of the fund moves, the value of Curits is insulated from this. Assuming that the value of funds raised far exceeds the funds required to build. \\

\noindent \textit{SGB: Perhaps these raised funds can be used to help stabilise the system in the same way that Basecoin does it. I.e. issue a small buyer of last resort function that perhaps takes a very small fee clip that adds to the collateral in reserve so that the system becomes more stable over time given the increasing reserves.}

\subsection{Approach}

\begin{itemize}
	\item Functionality of Curits and Nomins
	\item Initial Pricing of the value of the system:
		\subitem  1. market-found (likely to be undervalued and grow into your value)
		\subitem     2. pre-determined (\$5b)
		\subitem     3. Hybrid (capped ICO of a portion of the Curits at a value Haven specifies)
	\item 3. Ongoing pricing:
		\subitem     1. Dex - current proposal.
		\subitem     2. Rolling auction is still on the table.
		\subitem     3. Oracle
\end{itemize}

This section will provide an informal treatment of the proposed market's structure and dynamics, while section 2 will scaffold out its structure with a little more formality, including the definitions of all the important system variables. If you see an unfamiliar symbol in section 1, look up its definition in section 2.

\subsection{Investment incentives}

Why would anyone buy Curits in the first place? A potential buyer of Curits has at least three avenues for making money in Havven.

\paragraph{Capital gains due to the appreciation of Curits:}
Presumably the currency will appreciate due to a demand for Curits that is founded in the intrinsic utility of a stablecoin. Speculators will naturally be important players too.

\paragraph{Interest accrued from fees:}
If and when the price of Curits stabilises, then this may be the only long term positive-expected source of revenue. Ideally fees are set at a level where they are both high enough to be an incentive for rent-seekers to hold Curits in the long term (thus assuming the risk of providing collateral for the system) and low enough not to be a disincentive for ordinary users to transact in nomins.
It is desirable, perhaps in a future world dominated by micropayments, for these fees to be negligible for end users, while still being macroeconomically important for the system, and for those who capitalise it.

\paragraph{Arbitrage profit:}
It is the arbitrageurs who will ultimately bring the price of Nomins back into balance by a triangular circuit through Nomins, Curits, and the external (crypto or fiat) markets. They might hold Curits for a short time in order to pursue this strategy.


\pagebreak
\subsection{Fees}
There are a number of questions to be asked, and answered:
\begin{itemize}
    \item What are fees for?
    \item Who gets those fees?
    \item When can fees be levied?
    \item What macroeconomic effect does this levy have as a coin travels through the system?
\end{itemize}

\paragraph{The purpose of fees}

Fees will be redistributed to those who back the system, in order to incentivise people to capitalise it. The fee pool will be distributed periodically, for this purpose. However, if the system determines that the Nomin price is too low, then fees could be burned. If the price is too high then perhaps the system could sell these back into the system at a discounted rate.

Fee collection rate will also be a direct measure of the velocity of money in Havven. So it's in the interest of Curit holders to maximise liquidity in order to maximise their return.

\paragraph{Fee beneficiaries}

The previous assumption was that fees would simply be awarded to any holder of Curits. But then they get all the benefit with none of the risk. Although in the aggregate, it would be better for holders of Curits if everyone issued Nomins. The marginal return for any single player (who cannot issue a large fraction of all circulating Nomins) of actually issuing them would not outweigh the risk they take on in doing so. If a user can issue 1\% of circulating Nomins, then doing so will only increase their fee takings by 1\%.
Hence it it makes no sense to actually issue Nomins for any single player; so nobody will do it.

We must improve the marginal benefit of issuing Nomins into circulation in order to avoid this tragedy of the commons situation. So fees must be paid to those who issue Nomins, not just those who hold Curits.

\paragraph{Fee collection}

The system can potentially charge fees whenever any value is transferred, or any state is updated.

There are only a few circumstances that these things happen:

\begin{itemize}
    \item Nomin transfers
    \item Curit transfers
    \item Nomin issuance
    \item Curit redemption
\end{itemize}

The question is what levels to place these fees at. We might in general like to set higher Curit than Nomin transfer fees, making the stablecoin itself a lower friction market, in order to incentivise its use for exchange. Meanwhile, issuance and redemption fees will change the difficulty of entering and exiting the issuance game. \\

It's also possible for fees to float. The fee schedule could be altered dynamically in order to stabilise the system. It's even conceivable that the system could set negative fee rates if it needed to. We might charge punitive fees if a user is above the targeted utilisation ratio.

One example: if Nomin liquidity is low, meaning the system wants to incentivise issuance, then Nomin transfer fees could increase, thus having the combined effect of increasing the interest accrued by issuers, thus incentivising issuance and at the same time making it more expensive to transact in Nomins,
reducing demand and decreasing the liquidity requirements. \\

It should be pointed out that fees are antithetical to arbitrage. The higher the fee, the higher the friction, and the harder it is to make money by arbitrage. For example, if exchange fees amount to 1\% per trade, then a full arbitrage cycle between all three markets, (Nomins, Curits, and fiat) will cost in excess of 3\%. So it would not make sense to undertake arbitrage until such a time as the quoted exchange rate is misvalued by more than 3\% relative to the cross exchange rate. Hence, fees place a clear limit on the ability of arbitrage to stabilise price. Lower fees allow tighter stabilisation, within a window exactly in proportion with the fee rates themselves.


\pagebreak
\subsection{Encouraging liquidity}

\noindent It's desirable, when actors issue Nomins, that they are actually injected into the liquidity pool for their intended use, rather than be held by the same actor in order to benefit from both the receipt of fees but also the option of using those Nomins to release their Curits. The use of someone escrowing their Curits is that they provide backing for the currency flowing through the system, and so they should be rewarded for assuming this risk. In the fractional reserve system, this incentive is provided by the interest accrued upon the loans which generate money. It may be possible to adapt this system to Havven, by allowing issuers to escrow their Curits for a fixed time period, allowing the system to issue currency against that collateral, to be paid back a greater value of Nomins at a later time. \\

\noindent However, considering Havven as it stands today, there is an important question hanging over the mission of increasing the money supply. What's to stop someone issuing their Nomins, and then just holding onto them? In this manner they would accrue fees, but take on none of the risk of spending those Nomins, for they always have an instant option to liquidate their position and escape. An actor who had done the economically-desirable thing, on the other hand, who issued Nomins and then spent them, would be forced to buy Nomins in the open market in order to redeem their escrowed Curits. \\

\noindent If an issuer should not just be able to hold Nomins and accrue fees, that must also include letting them sit in another wallet they control. They should also not be able to sell their Nomins into the open market and with the proceeds buy the same value and let \textit{that} sit, only transferring it back to their main wallet once they want to flee.

\noindent But how to encourage a user to actually increase liquidity by buying goods with the Nomins they hold? 

\noindent Some level of Nomin liquidity above zero, where liquidity = flux = (average value).(average frequency). Consider also some optimum liquidity value above zero up to which diminishing returns are a factor.

The system may only ever be able to provide some level of stability within a set of thresholds (without actually backing the value of the token against the thing you're comparing it to, rocks, bottlecaps, USD.) This needs to be explicit.


\subsubsection{Non-discretionary Issuance}

One possibility is to simply provide an issuer no control over the tokens they issue. That is, when a quantity Nomins is issued, they are generated by the system, which then places a sell order at the current going rate for that quantity on an exchange on the behalf of the issuer. When the order is filled, the proceeds (in Curits, fiat, crypto, otherwise?) are remitted to the issuer.
Conversely, when a quantity of Nomins is burned, they must first be obtained from the open market. So a user would indicate an intention to burn, providing sufficient value to buy the proposed quantity of Nomins, and the system would bid for that quantity on their behalf, only liquidating the user's Curit position once they have been obtained.

\subsubsection{Motility}

But let us assume that we cannot force a user to issue and burn from the open market. We might like to encourage an issuer to spend their Nomins by other means. So we will give every account a motility score and pay fees in proportion with the product of this score and the number of tokens that account has issued.

This should be subject to common-sense obligations. The system should not be easily gamed. A user should not be able to cycle Nomins through accounts they control and collect fees. An issuer should not be able to just manipulate an account they control to have a high motility with small values and then dump a large value they want to hold into it. Ideally transferring value around repeatedly to manipulate the fee system would be expensive enough that the value lost to fees charged would outweigh the diminution of risk.

We would like to incentivise long transaction paths out of an account, and high out-degree nodes along those paths (so money is actually liquid/fungible).
We don't like short cycles. We don't like isolated subgraphs. Would be cool if the money could go into the main connected component of the transaction graph as quickly as possible, then circulate in there with high velocity.

\paragraph{Definitions}
\begin{align*}
    A \ &: \ \text{The set of all accounts} \\
    T \ &: \ \text{The multiset of all transactions; a subset of \(A \times A \times \mathbb{N}.\)} \\
    T_{a \rightarrow b} \subseteq T \ &: \ \text{The set of transactions from \(a\) to \(b\) with \(a, \ b \ \in A\).} \\
    v_t \ &: \ \text{the value of a transaction} \ t \in T. \\
    V_{a \rightarrow b} \ &:= \ \sum_{t \in T_{a \rightarrow b}}{v_t} \ \ \text{(the total value transferred from \(a\) to \(b\))} \\
    V_{a}^{in} \ &:= \ \sum_{p \in A}{V_{p \rightarrow a}} \\
    V_{a}^{out} \ &:= \ \sum_{p \in A}{V_{a \rightarrow p}} \\
    \intertext{We might interpret \((A, T)\) as a weighted multigraph of transactions, 
               with each transaction \(t \in T_{a \rightarrow b}\) corresponding to a weighted
               edge in that graph between nodes \(a\) and \(b\).
               Note that \(T_{a \rightarrow a} := \varnothing \), and hence \(V_{a \rightarrow a} = 0\)
               (accounts can't transfer to themselves).}
\end{align*}

\noindent We would like to know how likely a Nomin is to be spent soon from a given account.
The motility of the account should measure this. Considering an account \(a\), we will take
\(\mathcal{M}(a)\) to be the motility of \(a\):
\begin{align*}
    \mathcal{M}(a) &:= \sum_{p \in A}{P(a \ \text{transfers to} \ p) \cdot \mathcal{M}(p)} \\
    &= \sum_{p \in A}{\frac{V_{a \rightarrow p}}{V_{a}^{in}} \cdot \mathcal{M}(p)} \\
    &= \frac{1}{V_{a}^{in}} \sum_{p \in A}{V_{a \rightarrow p} \cdot \mathcal{M}(p)}
\end{align*}

Intuitively, if you transfer a lot of money to high-motility accounts, then your own motility is taken to be high.

\paragraph{Calculating Motility}
This will need to be calculated iteratively, and locally.
Note that \(V_{a \rightarrow p} = 0\) for \(p\) that \(a\) has never transferred to, so those accounts can be neglected.
It's probably too costly to store the value of \(V_{a \rightarrow b}\) explicitly. So we will have to eliminate this quantity in our expressions.
We will update motility scores whenever a new transaction \(t\) from \(a\) to \(b\) of value \(v_t\) is made.
\begin{align*}
    \intertext{Value into \(b\) increases, so \(\mathcal{M}(b)\) can be easily recalculated.} \\
    {V_{b}^{in}}' \ &\leftarrow \ V_{b}^{in} + v_t \\
    \mathcal{M}'(b) \ &\leftarrow \ \frac{1}{{V_{b}^{in}}'} \sum_{p \in A}{V_{b \rightarrow p} \cdot \mathcal{M}'(p)} \\
    \intertext{Meanwhile, the value transferred from \(a\) to \(b\) also increases.}
    V'_{a \rightarrow b} \ &\leftarrow \ V_{a \rightarrow b} + v_t \\
    \mathcal{M}'(a) \ &\leftarrow \ \frac{1}{V_{a}^{in}} \Big( V'_{a \rightarrow b} \cdot \mathcal{M}'(b) \ + \ \sum_{p \in A \backslash \{b\}}{V_{a \rightarrow p} \cdot \mathcal{M}'(p)}\Big)\\
    \intertext{Although these updates should also influence accounts which have (transitively) transferred into \(a\) and \(b\),
               we want to reward people for increasing liquidity today, rather than at some future time, and
               we take the motility of an account to be relatively stable after some time. As a result we will
               take \(\mathcal{M}'(p) \approx \mathcal{M}(p)\) for \(p \notin \{a, b\} \). Then:} 
    \mathcal{M}'(a) \ &\approx \ \frac{1}{V_{a}^{in}} \Big( (V_{a \rightarrow b} + v_t) \cdot \mathcal{M}'(b) \ + \ \sum_{p \in A \backslash \{b\}}{V_{a \rightarrow p} \cdot \mathcal{M}(p)}\Big) \\
    &\approx \ \frac{v_t}{V_{a}^{in}} \mathcal{M}'(b) \ + \frac{1}{V_{a}^{in}} \sum_{p \in A}{V_{a \rightarrow p} \cdot \mathcal{M}(p)} \\
    \intertext{So we will take our update step for a transaction \(t\) from \(a\) to \(b\) to be the following:}
    {V_{b}^{in}}' \ &\leftarrow \ V_{b}^{in} + v_t \\
    \mathcal{M}'(b) \ &\leftarrow \ \frac{V_{b}^{in}}{V_{b}^{in} + v_t} \mathcal{M}(b) \\
    \mathcal{M}'(a) \ &\leftarrow \ \frac{v_t}{V_{a}^{in}} \mathcal{M}'(b) \ + \mathcal{M}(a)
\end{align*}

\noindent It may also be nice to add a decay term, so that accounts that have not moved any money in a long time are taken to have a lower motility.

\subsection{System-actuated arbitrage}

It is agreed that the system relies upon arbitrage to bring the prices of its various components into equilibrium. Arbitrage is in some sense a form of economic friction, though. Arbitrageurs extract the quantity they skim as profit. Why not allow Havven itself to perform this arbitrage? This serves the dual purposes of improving the efficiency of the markets, and contributing to Havven's own capital pools.\\

\noindent In some imagined future, where Havven is its own blockchain, or has its own DEX, arbitrage opportunities could be spotted early, capitalised upon in a fee-free manner, and keeping all arbitrage profit within the havven ecosystem for productive price-stabilisation purposes, mining rewards, and any other uses that are found for this excess capital.

\pagebreak
\subsection{Utilisation ratio}

It's not clear to me exactly what purpose \(U_{max}\) serves. It certainly keeps the value of the pool of Curits below the value of the pool of Nomins, assuming there is no devaluation of a ratio more severe than \(U_{max}\) itself. However, if the system has adequate mechanisms enforce \(U\ \leq U_{max}\), then why not simply allow users to issue Nomins up to the maximum value of Curits they have escrowed? \\

\noindent A low \(U_{max}\) seems like it would place upward pressure on the price of Nomins. Consider a situation where \(U_{max} = 0.2\), and I have an impecunious friend, Jake, who owns a wallet which has issued \$20 worth of Nomins on the back of \$100 of escrowed Curits. At the moment, he has no
money, but his wallet is worth \$80, since he can burn \$20 worth of Nomins to get at those curits. So Jake should be willing to pay anywhere up to \$80 to buy enough Nomins to free up the Curits. This situation will still motivate Jake until the price of the Nomins he's issued is equal to the price of the Curits he's escrowed. That is, until the price of a Nomin is worth five times the price of a Curit.\\

\noindent Finally, let's consider the impact of the utilisation ratio on a Curit investor's value proposition. Examine the aggregate fees collected from Nomin transfers \(Ag_{nx}\), and expand out its definition:
\[Ag_{nx} = \frac{F_{nx} \cdot S_n \cdot C \cdot P_c \cdot U}{P_n}\]
This quantity is proportional with the actual utilisation ratio \(U\). The more Nomins that have been issued, the more fees are returned. So if \(U = 0.2\), then if the system would like to return a fee rate of 5\% per annum to Curit-holders, then fees to the tune of 25\% per year will have to be levied on Nomin transfers, assuming no other fees exist. This may be a little high.

\pagebreak
\subsection{Failure modes}
\subsubsection{Liquidity Trap}
\subsubsection{Trapped Currency}

\subsection{Assumptions}

\begin{enumerate}
	\item Ethereum will appropriately scale.
	\item Stability of existing cryptocurrencies will improve over time (declining utility of the stablecoin?).
	\item Unit of account will continue to be fiat for most use cases for the foreseeable future.
	\item The value of the platform is equal to the value of money raised in the ICO at that time.
   - the cost of developing the platform will be less than or equal to the amount raised.
	\item Price discovery - internal exchange, mirrors external exchanges (prices Curits in nomins, based on the value of nomins in USD).
	\item DEX follows external exchanges (arbitrage).
\end{enumerate}

\pagebreak
\section{Qualitative Scenario Analysis}

Havven is predicated on the notion that the money supply is 'backed' by the value of the reserve token in a similar fashion to a fractional reserve, however the fraction is the inverse of what has historically been implemented; there is always more 'value' in reserve than what has been issued as the medium of exchange.

\noindent Hence, the ratio of the backing value (in Curits) to the total supply of the exchange value (in Nomins) should not exceed some desired ceiling, enforced via incentives.

\subsection{Ratio moves favourably}

how can it move: ratio is made up of:
 number of nomins in circulation   x   nomin price.


2. Ratio moves unfavourably
   1. accumulate curits
      1. few curits available: what happens?
      2. no curits available: what happens?
   2. accumulate nomis
      1. few nomins available: what happens?
      2. no nomins available: what happens?





\pagebreak
\section{Quantitative System Analysis}

1. System Dynamics Modelling (sensitivity analysis, etc)
   1. method
   2. results



\hfill
\subsection{System variables}
\noindent What follows are the main variables of the system. 
Under each heading, each row will correspond to a single quantity of interest.
Each row will have three columns. Leftmost, a mathematical definition of the variable; in the middle, the dimension of the quantity (which units it is measured in); and on the rightmost, a short English summary of the variable.\\

\noindent Certain abbreviations will be used.
For example, \(\CUR{}\) and \(\NOM{}\) will be used as abbreviations for curits and nomins considered as units of measurement. \\

\paragraph{Prices}
\begin{align*}
    P_c & \ && &(\frac{\text{\$}}{\CUR{}}) && &\text{: curit price.} \\
    P_n & \ && &(\frac{\text{\$}}{\NOM{}}) && &\text{: nomin price.} \\
    \pi &:= \frac{P_c}{P_n} \ && &(\frac{\NOM{}}{\CUR{}}) && &\text{: curit to nomin conversion factor.} \\
    P_c' &= f(V_n, V_v) \cdot R && &(\frac{\text{\$}}{\NOM{} \cdot \text{sec}}) && &\text{: curit price rate of change.}
    \intertext{Here \(R\) is a risk term incoporating, for example, volatility, number of buyers versus sellers, and so on.}
\end{align*}
\\


\paragraph{Money Supply}
\begin{align*}
    &C \ && &(\CUR{}) && &\text{: Quantity of curits, which is constant.} \\
    &C_e \ && &(\CUR{}) && &\text{: Quantity of escrowed curits.} \\
    &N = C_N \cdot \pi \ && &(\NOM{}) && &\text{: Quantity of nomins. This can float.} \\
    &C_N = \frac{N}{\pi} \ && &(\CUR{}) && &\text{: Curit value of issued nomins.}
    \intertext{Ideally, \(C_N \leq C_e\).}
\end{align*}
\\

\paragraph{Utilisation Ratios}
\begin{align*}
    &U = \frac{C_N}{C} \ && &\text{(dimensionless)} && &\text{: Empirical issuance ratio. } \\
    &U_{max} \ && &\text{(dimensionless)} && &\text{: Targeted issuance ratio ceiling.}
    \intertext{Ideally, \(0 \leq U \leq U_{max} \leq 1\), but we need to work out a good level for \(U_{max}\).}
\end{align*}
\\

\paragraph{Microeconomic Variables} These should be defined as functions of \(P_n, \ P_c, \ \text{fees, etc.}\)
\begin{align*}
S_n \ && (\frac{1}{\text{sec}}) && &\text{: average nomin spend rate} \\
S_i \ && (\frac{1}{\text{sec}}) && &\text{: average issuance rate} \\
S_r \ && (\frac{1}{\text{sec}}) && &\text{: average redemption rate}
\end{align*}
\\

\paragraph{Money Movement}
\begin{align*}
    V_n &= S_n \cdot N \ && &(\frac{\NOM{}}{\text{sec}}) && &\text{: nomin transfer rate.} \\
    V_v &= V_i + V_r \ && &(\frac{\CUR{}}{\text{sec}}) && &\text{: nomin} \leftrightarrow \text{curit conversion rate.} \\
    V_i &= (C - C_N) \cdot S_i \ && &(\frac{\CUR{}}{\text{sec}}) && &\text{: nomin issuance rate.} \\
    V_r &= C_N \cdot S_r \ && &(\frac{\CUR{}}{\text{sec}}) && &\text{: curit redemption rate.} \\
    \intertext{\(V_i\) is assumed to grow as there are more free curits in the system.
               Actually perhaps it should grow with the number of escrowed curits with
               no nomins issued against them.}
    \intertext{\(V_r\), by contrast, is taken to grow proportionally with the number of escrowed curits.}
\end{align*}
\\

\paragraph{Fees}
\begin{align*}
\intertext{The following fees are ratios, for example 0.1\%, levied on each transaction.}
&F_{nx} & \ && &\text{(dimensionless)} && &\text{: nomin transfer fee} \\
&F_{cx} & \ && &\text{(dimensionless)} && &\text{: curit transfer fee} \\
&F_i & \ && &\text{(dimensionless)} && &\text{: nomin issuance fee} \\
&F_r & \ && &\text{(dimensionless)} && &\text{: curit redemption fee} \\
\intertext{These quantities are the aggregated fees accrued by the system per unit time.}
&Ag_{nx} &:= V_n \cdot F_{nx} \ && &(\frac{\NOM{}}{\text{sec}}) && &\text{: fees taken from nomin transfers.}
\end{align*}

\pagebreak
\section{Alternative approaches}

Outline the research to date done by Kain and why certain approaches were discarded.



\subsection{Basecoin}

\subsubsection{Description of system}

The Basecoin team appear to have mounted an admirable attempt to design a stablecoin, however we consider there to be a number of fatal issues in both the design and philosophy that are discussed below. \\

\noindent The whitepaper at the time of writing is still in draft, with much of it actually dedicated to explaining why a stable cryptocurrency would be useful. Only a high level description exists of how the stablisation mechanism operates. Basecoin is described as operating similarly to Havven in that there is separation between a backing token and a transactional token, however Basecoin also separates out a specific bond token. The peg to an arbitrary external asset is maintained by using an oracle service to discover the price on an external market, before regulating the supply of "basecoins" through actively increasing supply (issuing new basecoin), and decreasing supply (auctioning of bonds), effectively acting like an autonomous central bank.

\subsubsection{Key issues}

\noindent In the abstract, the paper indicates that Basecoin is "a cryptocurrency whose tokens can be robustly pegged to arbitrary assets or baskets of goods while remaining completely decentralized." While the system might run on a decentralised computing architecture, it is obviously inherently centralised due to the use of an oracle price-finding mechanism. The implementation and governance is also important for evaluating decentralisation, however no details are provided on this. Centralisation of price-finding is a key weakness in the Basecoin design. This weakness is recognised by the team in their discussion on how to implement price-finding in a decentralised fashion; whilst some discussion exists around various options for creating a pseudo-decentralised oracle, none are selected given that oracles by nature are centralised information bridges. The existence of a centralised attack vector is critically problematic. \\

\noindent Basecoin is intended to operate "as a decentralized, protocol-enforced algorithm, without the need for direct human judgment. For this reason, Basecoin can be understood as implementing an algorithmic central bank." Whilst not without merit, this approach was discarded by Havven due to the high degree of design complexity required to be anticipated in order to ensure the stabilisation mechanism is effective. The paper claims that Monte Carlo simulations have been run which indicate stability under a range of scenarios, however details are yet to be released by the team. Havven's model by contrast employs agent-based computational simulations to demonstrate the viability of the cryptoeconomic system. It is also far simpler in that the system is designed with only open market arbitrage incentives to encourage a stable peg. In this way, a set of rational participating actors can discover the price of the stablecoin rather than a single set of smart contracts that attempt to develop complex algorithms for processes that are today managed by a combination humans and markets. \\

\noindent In addition to the technical issues addressed there is an overarching philosophical difference in the approach taken by the Basecoin team which appears to be at odds with the consensus view of Bitcoin as an alternative to central bank monetary systems. In short, Basecoin attempts to replicate all of the features of central banking within a decentralised framework. We fundamentally disagree with any approach that can unilaterally manipulate the money supply. Central bank manipulation of the money supply is partially responsible for the boom and bust cycles that appear to be intractable at first glance but are in fact a direct result of governments restricting free markets from providing alternative forms of money through the enforcement of legal tender laws. \\ \\

\textit{... we need to discuss the above section. I've toned it down a little but it still appears to be pretty contentious. My view is, all we want to do is create a stablecoin vis a vis USD that a whole bunch of people use and believe in. If we can achieve that, we can then move onto more philosophical macroeconomical ironical aristotical discussions. }\\ \\

\noindent Another element not explored in the Basecoin whitepaper is the incentives for participants to engage with the cryptoeconomic system itself. While there is no argument against the utility of stablecoins, there must be incentives inherent in all such systems to ensure participation of all actors. In this case there are consumers of the stablecoin and active participants in the monetary policy. It is critical to be able to demonstrate that the incentives within the system will ensure profitable participation strategies for actors. Without this being clarified it is unclear as to whether there will be uptake by enough users to generate sufficient currency in circulation to support the demand for a stablecoin. Critically, the removal of Basecoin from the system to ensure the stable peg is predicated on the significant assumption that participants will take positions in the ongoing bond auctions. This assumption remains untested. \\
				
\noindent Some of the criticisms levelled at alternative stablecoins seem hypocritical, for example, "The only reason BitShares are worth 1 USD is because everyone believes it’ll be worth 1 USD." We would like to see the Basecoin team re-examine their understanding and/or clarify their description here relating to the fundamental nature of money, as this is the very thing that makes all money work: everyone believes it has value and will continue to do so. Without this critical element, all monetary systems fail. This can be due to poorly applied monetary policies in the failure of fiat currencies, or it can be inherent flaws in the monetary system itself; for example gold was replaced by paper backed by gold due to the improvements in transportability and security this new system provided. \\
				
\noindent Further, while a significant devaluation of Bitcoin (relative to say, USD) is a possibility, we feel that comparisons that imply that Bitcoin may experience structural and cyclical devaluation are unhelpful. The USD is inherently inflationary, and so some level of inflation of the stablecoin in order to maintain a peg is necessary. The problem with using an appreciating asset as money is that it acts to damp economic activity in times of economic stress and cause exacerbations of macroeconomic turmoil. This is precisely why the gold standard was abandoned in favour of fiat last century. Even though the Havven system is backed by an appreciating asset, it doesn't suffer from the same critical flaw as traditional representative money (or intrinsically-valuable money), because the two-currency system is designed to follow any external price, and in the case of following USD, we have a currency that is inherently inflationary (follows the USD inflation rate), and so Nomins denominated in USD would also be inflationary by the same rate. In this way, Havven can achieve its goal of operating as a interoperability technology between fiat and cryptocurrencies, in order to accelerate the adoption of this technology in a centrally-controlled and fiat-dominated world. \\

\noindent Of note, the whitepaper does not provide any implementation or performance considerations, including whether the system is intended to run on Ethereum or on a custom blockchain platform. This precipitates the obvious question regarding how such a decentralised system is paid for, as no mention is made within the whitepaper regarding levying fees or making use of fees to provide peg-supporting incentives. \\

\noindent A final point needs to be made with respect to the overarching monetary approach espoused in the whitepaper. In the section “Averting Macroeconomic Depressions” the authors appear to support money printing and inflationary policies and the subsequent devaluation of currency. Bitcoin and blockchain generally are often seen as anathema to centralised monetary systems, and we find it somewhat strange that the interventionist policies of quantitative easing (read vast money printing exercises) are praised in the paper. It seems there is a fundamental misapprehension of why fiat currency is still used at all given its long term tendency towards devaluation, it is solely due to legal tender laws, enforced through violence.  Quoting Hayek, “As one legal treatise on the law of money sums up the history of punishment for merely refusing to accept the legal money: ‘From Marco Polo we learn that, in the 13th century, Chinese law made the rejection of imperial paper money punishable by death, and twenty years in chains or, in some cases death, was the penalty provided for the refusal to accept French assignats. Early English law punished repudiation as lese-majesty. At the time of the American revolution, non-acceptance of Continental notes was treated as an enemy act and sometimes worked a forfeiture of the debt.” The idea that people would accept a decentralised monetary system that could arbitrarily impose a tax on savings in the form of inflation is hard to imagine. Even were it possible to demonstrate that inflation of the money supply via such a system would be effective in combating a deflationary spiral, a far better argument could be made that simply by implementing a stable store of value and unit of account that such a system would not be required. Generally, the apparent assumption that such a system would be achievable and still able to handle monetary crises in a far future time without centralised intervention stretches credulity. It's not entirely unclear why Basecoin has intended to merely replicate the function of a central bank, rather than aim for pure stability or a relative-stable approach such as Havven. \\

\noindent It is worth explicitly clarifying that we are skeptical of any group that would advocate for monetary approaches that are diametrically opposed to cryptoeconomic efforts to democratise money. Clearly, Bitcoin is not a perfect solution, but the proposal to intentionally create a systematically inflationary monetary system is not the answer. Instead, we should at this point in time be aiming to construct a system that provides a stable store of value relative to an arbitrary fiat currency. The macroeconomic benefits of such a system are clear and overwhelming, and for as long as we live in a fiat-dominated world this will be the case.

\subsubsection{Current state}

Basecoin at this stage have only released a draft whitepaper. A great number of implementation questions remain.

\subsection{Tether}

\subsubsection{Description of system}

Tethers accepts fiat deposits into the Hong Kong-based Tether Limited bank account and issues "USDT" (USD Tether) over Bitcoin via the Omni Layer protocol. Tethers are an asset-backed digital token, representing a claim on the cash held in reserve.

The stability of the USDT 'coin' effectively relies on the force of external market arbitrage to ensure the peg holds over time.

\subsubsection{Key issues}

Despite the whitepaper claiming that the "goal of any successful cryptocurrency is to completely eliminate the requirement for trust," and that each Tether is "fully redeemable/exchangeable any time for the underlying fiat currency," the company's terms of service quite clearly state that "there is no contractual right or other right or legal claim against us to redeem or exchange your Tethers for money."

Tether clearly relies on a manual, centralised proof of existence for the backing asset, and so suffers from the very issue that the Tether whitepaper decries. Indeed the same issue is encountered with tokenised gold, or similarly any other 'real-world' asset where some Oracle bridge is required to interface into a distributed ledger.

\subsubsection{Current state}

Recently, Tether announced support for issuing ERC-20 compatible tokens on Ethereum as opposed to releasing "tethers" on the Bitcoin blockchain using the Omni Layer protocol.

At the time of writing, the market capitalisation for USDT was approximately \$440m, and the discrepancy regarding their terms of service remains unresolved.


\subsection{MakerDAO}

\subsubsection{Description of system}

MakerDao has been around for a relatively long time in the pursuit of a stablecoin.

Complex system.

\subsubsection{Key issues}

\subsubsection{Current state}

Recently abandoned the Auction price-finding mechanism and are pursuing a 


\subsection{Nubits}

\subsubsection{Description of system}

\subsubsection{Key issues}

\subsubsection{Current state}



\pagebreak
\bibliography{citations.bib}
\bibliographystyle{plain}
\end{document}