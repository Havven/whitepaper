\section{Alternative approaches}

\subsection{Basecoin}

\paragraph{Description of system}

The Basecoin team appear to have mounted a somewhat a credible attempt to design a stablecoin, however we consider there to be a number of fatal issues that are discussed below. \\

\noindent The whitepaper at the time of writing is still in draft, with much of it actually dedicated to explaining why a stable cryptocurrency would be useful. Only a high level description exists of how the stablisation mechanism operates. Basecoin is described as operating similarly to Havven in that there is separation between a backing token and a transactional token, however Basecoin also separates out a specific ``bond'' token. The peg to an arbitrary external asset is maintained by using an oracle service to discover the price on an external market, before regulating the supply of ``basecoins'' through actively increasing (issuing new basecoin), and decreasing (auctioning of bonds) the supply, effectively acting as an autonomous central bank. \\

\subsubsection{Key issues}

\noindent In the abstract, the paper indicates that Basecoin is ``a cryptocurrency whose tokens can be robustly pegged to arbitrary assets or baskets of goods while remaining completely decentralized.'' While the system might run on a decentralised computing architecture, it is inherently centralised due to the use of the oracle price-finding mechanism. The implementation and governance is also important for evaluating decentralisation, however no details are provided. \\

\noindent Basecoin is intended to operate ``as a decentralized, protocol-enforced algorithm, without the need for direct human judgment (sic). For this reason, Basecoin can be understood as implementing an algorithmic central bank.'' Whilst not without merit, this approach was discarded by Havven due to the high degree of design complexity required to be anticipated in order to ensure the stabilisation mechanism is effective. The paper claims that Monte Carlo simulations have been run which indicate stability under a range of scenarios, however details are yet to be released by the team. Havven's model by contrast employs agent-based computational simulations to demonstrate the viability of the cryptoeconomic system. It is also far simpler in that the system is designed with only open market arbitrage incentives to encourage a stable peg. In this way, a set of rational participating actors can stabilise the price of the stablecoin rather than a single set of smart contracts that attempt to develop complex algorithms for processes that are today managed by a combination humans and markets. \\

\noindent Another element not explored in the Basecoin whitepaper is the incentives for participants to engage with the cryptoeconomic system itself. While there is no argument against the utility of stablecoins, there must be incentives inherent in all such systems to ensure the appropriate participation of all actors. In this case, there are consumers of the stablecoin and active participants in the monetary policy. It is critical to be able to demonstrate that the incentives within the system will ensure profitable participation strategies for actors. Without this being clarified it is unclear as to whether there will be uptake by enough users to generate sufficient currency in circulation to support the demand for a stablecoin. Critically, the removal of Basecoin from the system to ensure the stable peg is predicated on the significant assumption that participants will take positions in the ongoing bond auctions. This assumption remains untested. \\

\noindent Some of the criticisms levelled at alternative stablecoins seem hyper-critical; for example, ``The only reason BitShares are worth 1 USD is because everyone believes it’ll be worth 1 USD.'' We would like to see the Basecoin team re-examine their understanding and/or clarify their description here relating to the fundamental nature of money, as this is the very thing that makes all money work: everyone believes it has value and will continue to do so. Without this critical element, all monetary systems fail. This can be due to poorly applied monetary policies in the failure of fiat currencies, or it can be inherent flaws in the monetary system itself; for example gold was replaced by paper backed by gold due to the improvements in transportability and security this new system provided. \\

\noindent Further, while a significant devaluation of Bitcoin (relative to say, USD) is a possibility, we feel that comparisons that imply that Bitcoin may experience structural and cyclical devaluation are unhelpful. The USD is inherently inflationary, and so some level of inflation of the stablecoin in order to maintain a peg is necessary. The problem with using an appreciating asset as money is that it acts to damp economic activity in times of economic stress and cause exacerbations of macroeconomic turmoil. This is precisely why the gold standard was abandoned in favour of fiat last century. Even though the Havven system is backed by an appreciating asset, it doesn't suffer from the same critical flaw as traditional representative money (or intrinsically-valuable money), because the two-currency system is designed to follow any external price, and in the case of following USD, we have a currency that is inherently inflationary (follows the USD inflation rate), and so Nomins denominated in USD would also be inflationary by the same rate. In this way, Havven can achieve its goal of operating as a interoperability technology between fiat and cryptocurrencies and achieve its goal of accelerated adoption of this technology in a centrally-controlled and fiat-dominated world. \\

\noindent Of note, the whitepaper does not provide any implementation or performance considerations, including whether the system is intended to run on Ethereum or on a custom blockchain platform. This precipitates the obvious question regarding how such a decentralised system is paid for, as no mention is made within the whitepaper regarding levying fees or making use of fees to provide peg-supporting incentives. \\

\noindent A final point needs to be made with respect to the overarching monetary approach espoused in the whitepaper. In the section ``Averting Macroeconomic Depressions'' the authors appear to support money printing and inflationary policies and the subsequent devaluation of currency. Bitcoin and blockchain generally are often seen as anathema to centralised monetary systems, and we find it somewhat strange that the interventionist policies of quantitative easing (read vast money printing exercises) are praised in the paper. It seems there is a fundamental misapprehension of why fiat currency is still used at all given its long term tendency towards devaluation; it is solely due to legal tender laws enforced through the threat of violence. Hayek, quoting Nussbaum, ``As one legal treatise on the law of money sums up the history of punishment for merely refusing to accept the legal money: `From Marco Polo we learn that, in the 13th century, Chinese law made the rejection of imperial paper money punishable by death, and twenty years in chains or, in some cases death, was the penalty provided for the refusal to accept French assignats. Early English law punished repudiation as lese-majesty. At the time of the American revolution, non-acceptance of Continental notes was treated as an enemy act and sometimes worked a forfeiture of the debt.' ''~\cite{hayek1976denationalisation} The idea that people would accept a decentralised monetary system that could arbitrarily impose a tax on savings in the form of inflation is hard to imagine. Even were it possible to demonstrate that inflation of the money supply via such a system would be effective in combating a deflationary spiral, a far better argument could be made that simply by implementing a stable store of value and unit of account that such a system would not be required. Generally, the apparent assumption that such a system would be achievable and still able to handle monetary crises in a far future time without centralised intervention stretches credulity. It's not entirely unclear why Basecoin has intended to merely replicate the function of a central bank, rather than aim for pure stability or a relative-stable approach such as Havven. \\

\noindent It is worth explicitly clarifying that we are skeptical of any group that would advocate for monetary approaches that are diametrically opposed to cryptoeconomic efforts to democratise money. Clearly, Bitcoin is not a perfect solution, but the proposal to intentionally create a systematically inflationary monetary system is not the answer. Instead, we should at this point in time be aiming to construct a system that provides a stable store of value relative to an arbitrary fiat currency. The macroeconomic benefits of such a system are clear, and for as long as we live in a fiat-dominated world this will continue to be the case.

\subsubsection{Current state}

\paragraph{Key issues}

\paragraph{Current state}

\subsection{Tether}

\paragraph{Description of system}

Tethers accepts fiat deposits into the Hong Kong-based Tether Limited bank account and issues ``USDT'' (USD Tether) over Bitcoin via the Omni Layer protocol. Tethers are an asset-backed digital token, representing a claim on the cash held in reserve. \\

\noindent The stability of the USDT `coin' effectively relies on the force of external market arbitrage to ensure the peg holds over time. \\

\paragraph{Key issues}

Despite the whitepaper claiming that the ``goal of any successful cryptocurrency is to completely eliminate the requirement for trust,'' and that each Tether is ``fully redeemable/exchangeable any time for the underlying fiat currency,'' the company's terms of service quite clearly state that ``there is no contractual right or other right or legal claim against us to redeem or exchange your Tethers for money.'' \\

\noindent Tether clearly relies on a manual, centralised proof of existence for the backing asset, and so suffers from the very issue that the Tether whitepaper decries. Indeed the same issue is encountered with tokenised gold, or similarly any other real-world asset where some Oracle bridge is required to interface into a distributed ledger. \\

\paragraph{Current state}

Recently, Tether announced support for issuing ERC-20 compatible tokens on Ethereum as opposed to releasing ``tethers'' on the Bitcoin blockchain using the Omni Layer protocol. \\

\noindent At the time of writing, the market capitalisation for USDT was approximately \$440m, and the discrepancy regarding their terms of service remains unresolved. \\


\subsection{MakerDAO}

\paragraph{Description of system}

\paragraph{Key issues}

\paragraph{Current state}

\subsection{Nubits}

\paragraph{Description of system}

\paragraph{Key issues}

\paragraph{Current state}

\pagebreak