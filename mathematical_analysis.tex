\documentclass{article}
\usepackage[utf8]{inputenc}
\usepackage{mathtools}
\begin{document}

\section{Summary of System Variables}

\paragraph{Questions}
\begin{itemize}
\item Fees should only be given to those who have actually issued nomins?
\item How do we incentivise people to even transfer stuff.
\item Utilisation ratio needs to have distinct parts. Number of escrowed curits is not quite the same as the number of issued nomins.
\end{itemize}


\paragraph{Money Supply}
\begin{align*}
C & \ \text{(curits)} & \text{ : Quantity of curits, should be constant} \\
C_e &= C \cdot U \ \text{(curits)} & \text{ : Quantity of reserved curits, i.e. the value of tokens have been issued against } \\
N &= \frac{U_a \cdot C \cdot P_c}{P_n} \ \text{(nomins)} & \text{ : Quantity of nomins. This can float.}\\
\end{align*}
\\

\paragraph{Utilisation Ratios} We should work out a good level for $U_{max}$.
\begin{align*}
U &= \frac{P_n \cdot N}{C_e \cdot P_c} \ \text{(dimensionless)} & \text{ : Empirical issuance ratio. } \\
U_{max} & \text{(dimensionless)} & \text{ : Targeted issuance ratio ceiling. Ideally, } 0 \leq U \leq U_max \leq 1
\end{align*}
\\


\paragraph{Prices} These values are important, with the goal of stabilising the nomin price.
\begin{align*}
P_c & \ (\text{}\frac{\text{\$}}{\text{curit}}) & \text{ : curit price} \\
P_n & \ (\text{}\frac{\text{\$}}{\text{nomin}}) & \text{ : nomin price} \\
P_c' & = \alpha \cdot f(V_n, V_v) \cdot (\text{A risk term incorporating volatility? \#buyers - \#sellers?}) &
\end{align*}
\\


\paragraph{Fees}
\[F_x, F_i, F_r \ \text{(dimensionless) : transfer, issuance, redemption fees; these should be ratios, e.g. 0.1\%}\]
\\


\paragraph{Money Movement}
\begin{align*}
V_n &= S_n \cdot N \ (\frac{nomins}{second}) & \text{ : nomin transfer rate} \\
V_v &= V_i + V_r \ (\frac{curits}{second}) & \text{ : nomin <-> curit conversion rate.} \\
V_i &= (C - C_e) \cdot S_i \ (\frac{curits}{second}) & \text{ : nomin issuance rate. Assumed to grow as there are more free curits in the system (actually should probably grow with the number of escrowed but unissued nomins).} \\
V_r &= C_e \cdot S_r \ (\frac{curits}{second}) & \text{ : rate at which curits are redeemed in return for nomins (which are burned). Assumed to grow proportionally with the number of escrowed curits.}
\end{align*}
\\

\paragraph{Microeconomic Variables} These should be defined as functions of $P_n, P_c, \text{fees, etc.}$
\begin{align*}
& S_n \ \text{(1/second) : average nomin spend rate} \\
& S_i \ \text{(1/second) : average issuance rate} \\
& S_r \ \text{(1/second) : average redemption rate}
\end{align*}
\end{document}
